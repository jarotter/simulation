\documentclass[]{article}
\usepackage{lmodern}
\usepackage{amssymb,amsmath}
\usepackage{ifxetex,ifluatex}
\usepackage{fixltx2e} % provides \textsubscript
\ifnum 0\ifxetex 1\fi\ifluatex 1\fi=0 % if pdftex
  \usepackage[T1]{fontenc}
  \usepackage[utf8]{inputenc}
\else % if luatex or xelatex
  \ifxetex
    \usepackage{mathspec}
  \else
    \usepackage{fontspec}
  \fi
  \defaultfontfeatures{Ligatures=TeX,Scale=MatchLowercase}
\fi
% use upquote if available, for straight quotes in verbatim environments
\IfFileExists{upquote.sty}{\usepackage{upquote}}{}
% use microtype if available
\IfFileExists{microtype.sty}{%
\usepackage{microtype}
\UseMicrotypeSet[protrusion]{basicmath} % disable protrusion for tt fonts
}{}
\usepackage[margin=1in]{geometry}
\usepackage{hyperref}
\hypersetup{unicode=true,
            pdftitle={Tarea 3},
            pdfauthor={Sergio Arnaud, Jorge Rotter},
            pdfborder={0 0 0},
            breaklinks=true}
\urlstyle{same}  % don't use monospace font for urls
\usepackage{color}
\usepackage{fancyvrb}
\newcommand{\VerbBar}{|}
\newcommand{\VERB}{\Verb[commandchars=\\\{\}]}
\DefineVerbatimEnvironment{Highlighting}{Verbatim}{commandchars=\\\{\}}
% Add ',fontsize=\small' for more characters per line
\usepackage{framed}
\definecolor{shadecolor}{RGB}{248,248,248}
\newenvironment{Shaded}{\begin{snugshade}}{\end{snugshade}}
\newcommand{\AlertTok}[1]{\textcolor[rgb]{0.94,0.16,0.16}{#1}}
\newcommand{\AnnotationTok}[1]{\textcolor[rgb]{0.56,0.35,0.01}{\textbf{\textit{#1}}}}
\newcommand{\AttributeTok}[1]{\textcolor[rgb]{0.77,0.63,0.00}{#1}}
\newcommand{\BaseNTok}[1]{\textcolor[rgb]{0.00,0.00,0.81}{#1}}
\newcommand{\BuiltInTok}[1]{#1}
\newcommand{\CharTok}[1]{\textcolor[rgb]{0.31,0.60,0.02}{#1}}
\newcommand{\CommentTok}[1]{\textcolor[rgb]{0.56,0.35,0.01}{\textit{#1}}}
\newcommand{\CommentVarTok}[1]{\textcolor[rgb]{0.56,0.35,0.01}{\textbf{\textit{#1}}}}
\newcommand{\ConstantTok}[1]{\textcolor[rgb]{0.00,0.00,0.00}{#1}}
\newcommand{\ControlFlowTok}[1]{\textcolor[rgb]{0.13,0.29,0.53}{\textbf{#1}}}
\newcommand{\DataTypeTok}[1]{\textcolor[rgb]{0.13,0.29,0.53}{#1}}
\newcommand{\DecValTok}[1]{\textcolor[rgb]{0.00,0.00,0.81}{#1}}
\newcommand{\DocumentationTok}[1]{\textcolor[rgb]{0.56,0.35,0.01}{\textbf{\textit{#1}}}}
\newcommand{\ErrorTok}[1]{\textcolor[rgb]{0.64,0.00,0.00}{\textbf{#1}}}
\newcommand{\ExtensionTok}[1]{#1}
\newcommand{\FloatTok}[1]{\textcolor[rgb]{0.00,0.00,0.81}{#1}}
\newcommand{\FunctionTok}[1]{\textcolor[rgb]{0.00,0.00,0.00}{#1}}
\newcommand{\ImportTok}[1]{#1}
\newcommand{\InformationTok}[1]{\textcolor[rgb]{0.56,0.35,0.01}{\textbf{\textit{#1}}}}
\newcommand{\KeywordTok}[1]{\textcolor[rgb]{0.13,0.29,0.53}{\textbf{#1}}}
\newcommand{\NormalTok}[1]{#1}
\newcommand{\OperatorTok}[1]{\textcolor[rgb]{0.81,0.36,0.00}{\textbf{#1}}}
\newcommand{\OtherTok}[1]{\textcolor[rgb]{0.56,0.35,0.01}{#1}}
\newcommand{\PreprocessorTok}[1]{\textcolor[rgb]{0.56,0.35,0.01}{\textit{#1}}}
\newcommand{\RegionMarkerTok}[1]{#1}
\newcommand{\SpecialCharTok}[1]{\textcolor[rgb]{0.00,0.00,0.00}{#1}}
\newcommand{\SpecialStringTok}[1]{\textcolor[rgb]{0.31,0.60,0.02}{#1}}
\newcommand{\StringTok}[1]{\textcolor[rgb]{0.31,0.60,0.02}{#1}}
\newcommand{\VariableTok}[1]{\textcolor[rgb]{0.00,0.00,0.00}{#1}}
\newcommand{\VerbatimStringTok}[1]{\textcolor[rgb]{0.31,0.60,0.02}{#1}}
\newcommand{\WarningTok}[1]{\textcolor[rgb]{0.56,0.35,0.01}{\textbf{\textit{#1}}}}
\usepackage{graphicx,grffile}
\makeatletter
\def\maxwidth{\ifdim\Gin@nat@width>\linewidth\linewidth\else\Gin@nat@width\fi}
\def\maxheight{\ifdim\Gin@nat@height>\textheight\textheight\else\Gin@nat@height\fi}
\makeatother
% Scale images if necessary, so that they will not overflow the page
% margins by default, and it is still possible to overwrite the defaults
% using explicit options in \includegraphics[width, height, ...]{}
\setkeys{Gin}{width=\maxwidth,height=\maxheight,keepaspectratio}
\IfFileExists{parskip.sty}{%
\usepackage{parskip}
}{% else
\setlength{\parindent}{0pt}
\setlength{\parskip}{6pt plus 2pt minus 1pt}
}
\setlength{\emergencystretch}{3em}  % prevent overfull lines
\providecommand{\tightlist}{%
  \setlength{\itemsep}{0pt}\setlength{\parskip}{0pt}}
\setcounter{secnumdepth}{0}
% Redefines (sub)paragraphs to behave more like sections
\ifx\paragraph\undefined\else
\let\oldparagraph\paragraph
\renewcommand{\paragraph}[1]{\oldparagraph{#1}\mbox{}}
\fi
\ifx\subparagraph\undefined\else
\let\oldsubparagraph\subparagraph
\renewcommand{\subparagraph}[1]{\oldsubparagraph{#1}\mbox{}}
\fi

%%% Use protect on footnotes to avoid problems with footnotes in titles
\let\rmarkdownfootnote\footnote%
\def\footnote{\protect\rmarkdownfootnote}

%%% Change title format to be more compact
\usepackage{titling}

% Create subtitle command for use in maketitle
\newcommand{\subtitle}[1]{
  \posttitle{
    \begin{center}\large#1\end{center}
    }
}

\setlength{\droptitle}{-2em}

  \title{Tarea 3}
    \pretitle{\vspace{\droptitle}\centering\huge}
  \posttitle{\par}
    \author{Sergio Arnaud, Jorge Rotter}
    \preauthor{\centering\large\emph}
  \postauthor{\par}
      \predate{\centering\large\emph}
  \postdate{\par}
    \date{10/11/2018}

\usepackage{amsmath}
\usepackage{mdsymbol}

\begin{document}
\maketitle

\hypertarget{pregunta-1}{%
\subsection{Pregunta 1}\label{pregunta-1}}

Un estadístico está interesado en el número N de peces en un estanque.
El captura 250 peces, los marca y los regresa al estanque. Unos cuantos
días después regresa y atrapa peces hasta que obtiene 50 peces marcados,
en ese punto también tiene 124 peces no marcados (la muestra total es de
174 peces).

\begin{itemize}
\tightlist
\item
  ¿Cuál es la estimación de N?
\end{itemize}

\(\frac{124}{50} = 2.48\) de forma que la estimación de N está dada por
\(2.48\cdot250=620\)

\begin{itemize}
\tightlist
\item
  Hagan un programa (en excel o en R), que permita simular el proceso de
  obtener la primera y segunda muestra considerando como parámetros el
  tamaño N de la población de interés, el tamaño de la primera y segunda
  muestra y como datos a estimar son: de qué tamaño debe ser n1 y n2
  para obtener una buena aproximación y ver cómo se afecta por el tamaño
  N.
\end{itemize}

El siguiente código, para una N dada (en este caso N=620) regresa los
tamaños de muestra (n1 y n2) que dan la mejor estimación de N donde n1
es la muestra que se debe de obtener y marcar en una primera instancia y
n2 es la muestra que se debe obtener en una segunda instancia para
comparar los marcados vs los no marcados y estimar la población total.

\begin{Shaded}
\begin{Highlighting}[]
\NormalTok{N <-}\StringTok{ }\DecValTok{620}
\NormalTok{j <-}\DecValTok{1}
\NormalTok{means <-}\KeywordTok{matrix}\NormalTok{(}\DecValTok{0}\NormalTok{, }\DataTypeTok{nrow=}\KeywordTok{round}\NormalTok{(N}\OperatorTok{/}\DecValTok{4}\NormalTok{), }\DataTypeTok{ncol=}\KeywordTok{round}\NormalTok{(N}\OperatorTok{/}\DecValTok{4}\NormalTok{))}
\ControlFlowTok{for}\NormalTok{ (N1 }\ControlFlowTok{in} \DecValTok{1}\OperatorTok{:}\KeywordTok{round}\NormalTok{(N}\OperatorTok{/}\DecValTok{4}\NormalTok{))\{}
  \ControlFlowTok{for}\NormalTok{ (N2 }\ControlFlowTok{in} \DecValTok{1}\OperatorTok{:}\KeywordTok{round}\NormalTok{(N}\OperatorTok{/}\DecValTok{4}\NormalTok{))\{}
\NormalTok{    aprox =}\StringTok{ }\KeywordTok{rep}\NormalTok{(}\DecValTok{0}\NormalTok{,}\DecValTok{15}\NormalTok{)}
    \ControlFlowTok{for}\NormalTok{ (i }\ControlFlowTok{in} \DecValTok{1}\OperatorTok{:}\DecValTok{15}\NormalTok{)\{}
\NormalTok{       l =}\StringTok{ }\KeywordTok{c}\NormalTok{(}\KeywordTok{rep}\NormalTok{(}\DecValTok{0}\NormalTok{,N1),}\KeywordTok{rep}\NormalTok{(}\DecValTok{1}\NormalTok{,N}\OperatorTok{-}\NormalTok{N1))}
\NormalTok{       s <-}\StringTok{ }\KeywordTok{sample}\NormalTok{(l,N2)}
\NormalTok{       aprox[i] =}\StringTok{ }\NormalTok{(}\KeywordTok{sum}\NormalTok{(s)}\OperatorTok{/}\NormalTok{(}\KeywordTok{length}\NormalTok{(s)}\OperatorTok{-}\KeywordTok{sum}\NormalTok{(s)))}\OperatorTok{*}\NormalTok{N1}
\NormalTok{    \}}
\NormalTok{    means[[N1,N2]] =}\StringTok{ }\KeywordTok{mean}\NormalTok{(aprox)}
\NormalTok{    j =}\StringTok{ }\NormalTok{j}\OperatorTok{+}\DecValTok{1}
\NormalTok{  \}}
\NormalTok{\}}
\NormalTok{ans <-}\StringTok{ }\KeywordTok{which}\NormalTok{((}\KeywordTok{abs}\NormalTok{(means }\OperatorTok{-}\StringTok{ }\NormalTok{N)) }\OperatorTok{==}\StringTok{ }\KeywordTok{min}\NormalTok{(}\KeywordTok{abs}\NormalTok{(means }\OperatorTok{-}\StringTok{ }\NormalTok{N)),}\DataTypeTok{arr.ind=}\NormalTok{T)}
\CommentTok{# n1}
\NormalTok{ans[}\DecValTok{1}\NormalTok{]}
\end{Highlighting}
\end{Shaded}

\begin{verbatim}
## [1] 36
\end{verbatim}

\begin{Shaded}
\begin{Highlighting}[]
\CommentTok{# n2}
\NormalTok{ans[}\DecValTok{2}\NormalTok{]}
\end{Highlighting}
\end{Shaded}

\begin{verbatim}
## [1] 94
\end{verbatim}

\hypertarget{pregunta-2}{%
\subsection{Pregunta 2}\label{pregunta-2}}

Este problema es una versión simplificada de dos problemas comunes que
enfretan las compañías de seguros: calcular la probabilidad de quebrar y
estimar cuánto dinero podrán hacer. Supongan que una compañía tiene
activos (todo en dólares) por \(\$10^6\) y \(n=1,000\) clientes que
pagan individualmente una prima anual de \$5,500 al principio de cada
año. Basándose en experiencia previa, se estima que la probabilidad de
que un cliente haga reclamo es \(p=0.1\) por año, independientemente de
reclamos previos de otros clientes. El tamaño \(X\) de los reclamos
varía, y tiene la siguiente densidad con \(\alpha = 5\) y
\(\beta = 125,000\) \[
f(x) = I(x\geq0)\frac{\alpha \beta^\alpha}{(x+\beta)^{\alpha+1}}
\]

Consideremos la fortuna de la compañía en un horizonte de cinco años, y
sea \(Z(t)\) la cantidad de activo al terminar el año \(t\), de manera
que \(Z(0)=1,000,000\) y
\(Z(t)=I(Z(t-1)>0)\max\{Z(t-1)+\textrm{primas}-\textrm{reclamos},0\}\)

\begin{enumerate}
\def\labelenumi{\alph{enumi}.}
\tightlist
\item
  Calcule la función de distribución \(F_X\), \(\mathbb{E}[X]\)y
  \(\textrm{Var}(X)\). Obtenga por simulación una muestra de \(X\),
  calcule su distribución empírica y compare con la verdadera.
\end{enumerate}

\[
\begin{align}
F_X(x) &= \int_0^x \frac{\alpha \beta^\alpha}{(t+\beta)^{\alpha+1}}dt \\
&= \int_\beta^{x+\beta}\alpha \beta^\alpha u^{-(\alpha+1)} du\\
&= -\beta^\alpha \left[ \frac{1}{u^\alpha} \right]^{x+\beta}_\beta \\ 
&= 1 - \left(1+\frac{\beta}{x+\beta} \right)^\alpha
\end{align}
\]

Para la esperanza y varianza, notemos que \[
\begin{align}
\mathbb{E}[(X + \beta)^k] &= \int_0^\infty\frac{(x+b)^k\alpha\beta^\alpha}{(x+\beta)^{\alpha+1}} \\
&= \int_0^\infty \frac{\alpha \beta^\alpha}{(\beta+x)^{\alpha + 1 - k}} dx \\
&= \int_0^\infty\frac{\alpha \beta^k}{\alpha-k} \frac{(\alpha-k) \beta^{\alpha-k}}{(\beta+x)^{\alpha-k+1}} dx \\
&= \frac{\alpha \beta^k}{\alpha-k} \int_0^\infty f_Y(x) dx \\
&=\frac{\alpha \beta^k}{\alpha-k}
\end{align}
\]

Donde \(Y~Pareto(\alpha-k, \beta)\) (la misma familia que \(X\)).

Luego,

\[
\begin{align}
\mathbb{E}[X] &=  \mathbb{E}[X+\beta]-\beta \\
&= \frac{\alpha\beta}{\alpha-1} - \beta  \\
&= \frac{\beta}{\alpha-1}
\end{align}
\]

\[
\begin{align}
\mathrm{Var}(X) &= \mathbb{E}[X^2]-\mathbb{E}[X]^2 \\
&=  \mathbb{E}[(X+\beta)^2] - 2\beta\mathbb{E}[X] - \beta^2 - \mathbb{E}[X]^2 \\ 
&=\frac{\alpha \beta^2}{\alpha-2} - \frac{2\beta^2}{\alpha-1} - \beta^2 -\left( \frac{\beta}{\alpha-1}\right)^2 \\ 
&= \frac{\alpha \beta^2}{(\alpha-1)^2(\alpha-2)}
\end{align}
\]

Y para muestrear, usamos el método de la transformación inversa \[
\begin{align}
y=1-\left( \frac{\beta}{x+\beta} \right)^\alpha &\Leftrightarrow 1-y=\left( \frac{\beta}{x+\beta} \right)^\alpha \\
&\Leftrightarrow (1-y)^{\frac{1}{\alpha}} = \frac{\beta}{x+\beta} \\
&\Leftrightarrow \beta (1-y)^{-\frac{1}{\alpha}} - \beta = x
\end{align}
\]

\begin{Shaded}
\begin{Highlighting}[]
\NormalTok{qpareto <-}\StringTok{ }\ControlFlowTok{function}\NormalTok{(x, alpha, beta)\{}
\NormalTok{  beta}\OperatorTok{*}\NormalTok{((}\DecValTok{1}\OperatorTok{-}\NormalTok{x)}\OperatorTok{^}\NormalTok{(}\OperatorTok{-}\DecValTok{1}\OperatorTok{/}\NormalTok{alpha) }\OperatorTok{-}\StringTok{ }\DecValTok{1}\NormalTok{)}
\NormalTok{\}}

\NormalTok{ppareto <-}\StringTok{ }\ControlFlowTok{function}\NormalTok{(x, alpha, beta)\{}
  \DecValTok{1}\OperatorTok{-}\NormalTok{(}\DecValTok{1}\OperatorTok{+}\NormalTok{x}\OperatorTok{/}\NormalTok{beta)}\OperatorTok{^}\NormalTok{(}\OperatorTok{-}\NormalTok{alpha)}
\NormalTok{\}}

\NormalTok{rpareto <-}\StringTok{ }\ControlFlowTok{function}\NormalTok{(n, alpha, beta)\{}
\NormalTok{  u <-}\StringTok{ }\KeywordTok{runif}\NormalTok{(n)}
  \KeywordTok{qpareto}\NormalTok{(u, alpha, beta)}
\NormalTok{\}}

\NormalTok{x_sample <-}\StringTok{ }\KeywordTok{data_frame}\NormalTok{(}\DataTypeTok{simulado=}\KeywordTok{rpareto}\NormalTok{(}\DecValTok{1000}\NormalTok{,}\DecValTok{5}\NormalTok{, }\DecValTok{125000}\NormalTok{))}
  
\KeywordTok{ggplot}\NormalTok{() }\OperatorTok{+}
\StringTok{  }\KeywordTok{stat_ecdf}\NormalTok{(}\DataTypeTok{data=}\NormalTok{x_sample, }
            \DataTypeTok{mapping=}\KeywordTok{aes}\NormalTok{(}\DataTypeTok{x=}\NormalTok{simulado, }\DataTypeTok{colour=}\StringTok{'simulado'}\NormalTok{), }
            \DataTypeTok{geom=}\StringTok{'step'}\NormalTok{, }
            \DataTypeTok{pad =}\NormalTok{ F) }\OperatorTok{+}
\StringTok{  }\KeywordTok{stat_function}\NormalTok{(}\DataTypeTok{data=}\KeywordTok{data_frame}\NormalTok{(}\DataTypeTok{mx=}\KeywordTok{c}\NormalTok{(}\DecValTok{0}\NormalTok{,}\KeywordTok{max}\NormalTok{(x_sample}\OperatorTok{$}\NormalTok{simulado))),}
                \DataTypeTok{mapping=}\KeywordTok{aes}\NormalTok{(}\DataTypeTok{colour=}\StringTok{'teórico'),}
\StringTok{                fun=function(x) ppareto(x,5,125000))+}
\StringTok{  labs(x='}\NormalTok{x}\StringTok{', y='}\KeywordTok{F}\NormalTok{(x)}\StringTok{')}
\end{Highlighting}
\end{Shaded}

\includegraphics{sim-t3-sol_files/figure-latex/unnamed-chunk-3-1.pdf}

\begin{Shaded}
\begin{Highlighting}[]
\KeywordTok{ks.test}\NormalTok{(x_sample}\OperatorTok{$}\NormalTok{simulado, }\StringTok{'ppareto'}\NormalTok{, }\DataTypeTok{alpha=}\DecValTok{5}\NormalTok{, }\DataTypeTok{beta=}\DecValTok{125000}\NormalTok{)}
\end{Highlighting}
\end{Shaded}

\begin{verbatim}
## 
##  One-sample Kolmogorov-Smirnov test
## 
## data:  x_sample$simulado
## D = 0.02736, p-value = 0.4425
## alternative hypothesis: two-sided
\end{verbatim}

\begin{Shaded}
\begin{Highlighting}[]
\KeywordTok{rm}\NormalTok{(x_sample)}
\end{Highlighting}
\end{Shaded}

La prueba no da evidencia para rechazar la hipótesis nula, y en la
gráfica sí se ven similares ambas distribuciones.

Escriba una función para simular los activos de la compañía por cinco
años

\begin{Shaded}
\begin{Highlighting}[]
\NormalTok{activos <-}\StringTok{ }\ControlFlowTok{function}\NormalTok{(activo_inicial, n_clientes, prima, p, horizonte, alpha, beta)\{}

\NormalTok{  z <-}\StringTok{ }\KeywordTok{rep}\NormalTok{(}\DecValTok{0}\NormalTok{,horizonte}\OperatorTok{+}\DecValTok{1}\NormalTok{)}
\NormalTok{  z[}\DecValTok{1}\NormalTok{] <-}\StringTok{ }\NormalTok{activo_inicial}

  \ControlFlowTok{for}\NormalTok{(i }\ControlFlowTok{in} \DecValTok{2}\OperatorTok{:}\NormalTok{(horizonte}\OperatorTok{+}\DecValTok{1}\NormalTok{))\{}

    \ControlFlowTok{if}\NormalTok{(z[i}\DecValTok{-1}\NormalTok{] }\OperatorTok{<=}\StringTok{ }\DecValTok{0}\NormalTok{) }\ControlFlowTok{break}

\NormalTok{    reclamos <-}\StringTok{ }\KeywordTok{rbernoulli}\NormalTok{(n_clientes, p)}
\NormalTok{    montos <-}\StringTok{ }\KeywordTok{rpareto}\NormalTok{(n_clientes, alpha, beta)}\OperatorTok{*}\NormalTok{reclamos}
\NormalTok{    z[i] <-}\StringTok{ }\KeywordTok{max}\NormalTok{(z[i}\DecValTok{-1}\NormalTok{]}\OperatorTok{+}\NormalTok{n_clientes}\OperatorTok{*}\NormalTok{prima}\OperatorTok{-}\KeywordTok{sum}\NormalTok{(montos), }\DecValTok{0}\NormalTok{)}
\NormalTok{  \}}
  
\NormalTok{  z}
\NormalTok{\}}

\NormalTok{activos_n<-}\StringTok{ }\ControlFlowTok{function}\NormalTok{(n, }\DataTypeTok{n_clientes=}\DecValTok{1000}\NormalTok{, }
                     \DataTypeTok{activo_inicial=}\FloatTok{1e6}\NormalTok{, }\DataTypeTok{prima=}\DecValTok{5500}\NormalTok{, }\DataTypeTok{p=}\FloatTok{0.1}\NormalTok{,}
                     \DataTypeTok{alpha=}\DecValTok{5}\NormalTok{, }\DataTypeTok{beta=}\DecValTok{125000}\NormalTok{, }
                     \DataTypeTok{horizonte=}\DecValTok{5}\NormalTok{)\{}
  \KeywordTok{replicate}\NormalTok{(n, }\KeywordTok{activos}\NormalTok{(activo_inicial, n_clientes, prima, p, horizonte, alpha, beta)) }\OperatorTok
\StringTok{    }\KeywordTok{t}\NormalTok{()}
\NormalTok{\}}
\end{Highlighting}
\end{Shaded}

Estime la probabilidad de que la compañía se vaya a bancarrota

\begin{Shaded}
\begin{Highlighting}[]
\KeywordTok{mean}\NormalTok{(}\KeywordTok{activos_n}\NormalTok{(}\DecValTok{10000}\NormalTok{)[,}\DecValTok{6}\NormalTok{]}\OperatorTok{==}\DecValTok{0}\NormalTok{)}
\end{Highlighting}
\end{Shaded}

\begin{verbatim}
## [1] 1e-04
\end{verbatim}

Y la ganancia esperada

\begin{Shaded}
\begin{Highlighting}[]
\KeywordTok{mean}\NormalTok{(}\KeywordTok{activos_n}\NormalTok{(}\DecValTok{10000}\NormalTok{)[,}\DecValTok{6}\NormalTok{])}
\end{Highlighting}
\end{Shaded}

\begin{verbatim}
## [1] 12860946
\end{verbatim}

Suponga ahora que la compañía toma ganancias. Repita las preguntas
anteriores.

\begin{Shaded}
\begin{Highlighting}[]
\NormalTok{activos_gan <-}\StringTok{ }\ControlFlowTok{function}\NormalTok{(activo_inicial, n_clientes, prima, p, horizonte, alpha, beta)\{}

\NormalTok{  z <-}\StringTok{ }\KeywordTok{rep}\NormalTok{(}\DecValTok{0}\NormalTok{,horizonte}\OperatorTok{+}\DecValTok{1}\NormalTok{)}
\NormalTok{  ganancias <-}\StringTok{ }\DecValTok{0}
\NormalTok{  z[}\DecValTok{1}\NormalTok{] <-}\StringTok{ }\NormalTok{activo_inicial}

  \ControlFlowTok{for}\NormalTok{(i }\ControlFlowTok{in} \DecValTok{2}\OperatorTok{:}\NormalTok{(horizonte}\OperatorTok{+}\DecValTok{1}\NormalTok{))\{}

    \ControlFlowTok{if}\NormalTok{(z[i}\DecValTok{-1}\NormalTok{] }\OperatorTok{<=}\StringTok{ }\DecValTok{0}\NormalTok{) }\ControlFlowTok{break}

\NormalTok{    reclamos <-}\StringTok{ }\KeywordTok{rbernoulli}\NormalTok{(n_clientes, p)}
\NormalTok{    montos <-}\StringTok{ }\KeywordTok{rpareto}\NormalTok{(n_clientes, alpha, beta)}\OperatorTok{*}\NormalTok{reclamos}

\NormalTok{    z[i] <-}\StringTok{ }\NormalTok{z[i}\DecValTok{-1}\NormalTok{] }\OperatorTok{+}\StringTok{ }\NormalTok{n_clientes}\OperatorTok{*}\NormalTok{prima }\OperatorTok{-}\StringTok{ }\KeywordTok{sum}\NormalTok{(montos)}
    
    \ControlFlowTok{if}\NormalTok{(z[i] }\OperatorTok{>}\StringTok{ }\NormalTok{activo_inicial)\{}
\NormalTok{      ganancias <-}\StringTok{ }\NormalTok{ganancias }\OperatorTok{+}\StringTok{ }\NormalTok{z[i] }\OperatorTok{-}\StringTok{ }\NormalTok{activo_inicial}
\NormalTok{      z[i] <-}\StringTok{ }\NormalTok{activo_inicial}
\NormalTok{    \}}
\NormalTok{  \}}
  
\NormalTok{  z}
\NormalTok{\}}

\NormalTok{activos_gan_n<-}\StringTok{ }\ControlFlowTok{function}\NormalTok{(n, }\DataTypeTok{n_clientes=}\DecValTok{1000}\NormalTok{, }
                     \DataTypeTok{activo_inicial=}\FloatTok{1e6}\NormalTok{, }\DataTypeTok{prima=}\DecValTok{5500}\NormalTok{, }\DataTypeTok{p=}\FloatTok{0.1}\NormalTok{,}
                     \DataTypeTok{alpha=}\DecValTok{5}\NormalTok{, }\DataTypeTok{beta=}\DecValTok{125000}\NormalTok{, }
                     \DataTypeTok{horizonte=}\DecValTok{5}\NormalTok{)\{}
  \KeywordTok{replicate}\NormalTok{(n, }\KeywordTok{activos_gan}\NormalTok{(activo_inicial, n_clientes, prima, p, horizonte, alpha, beta)) }\OperatorTok
\StringTok{    }\KeywordTok{t}\NormalTok{()}
\NormalTok{\}}

\NormalTok{nueva_forma <-}\StringTok{ }\KeywordTok{activos_gan_n}\NormalTok{(}\DecValTok{1000}\NormalTok{)}

\KeywordTok{mean}\NormalTok{(nueva_forma[,}\DecValTok{6}\NormalTok{])}
\end{Highlighting}
\end{Shaded}

\begin{verbatim}
## [1] 1e+06
\end{verbatim}

\begin{Shaded}
\begin{Highlighting}[]
\KeywordTok{mean}\NormalTok{(nueva_forma[,}\DecValTok{6}\NormalTok{]}\OperatorTok{==}\DecValTok{0}\NormalTok{)}
\end{Highlighting}
\end{Shaded}

\begin{verbatim}
## [1] 0
\end{verbatim}

\hypertarget{pregunta-3}{%
\subsection{Pregunta 3}\label{pregunta-3}}

Las densidades dadas son:

\begin{Shaded}
\begin{Highlighting}[]
\NormalTok{cauchy <-}\StringTok{ }\ControlFlowTok{function}\NormalTok{(x, beta, gamma)\{ }
  \DecValTok{1} \OperatorTok{/}\StringTok{ }\NormalTok{(pi}\OperatorTok{*}\NormalTok{beta}\OperatorTok{*}\NormalTok{(}\DecValTok{1} \OperatorTok{+}\StringTok{ }\NormalTok{((x}\OperatorTok{-}\NormalTok{gamma)}\OperatorTok{/}\NormalTok{beta)}\OperatorTok{^}\DecValTok{2}\NormalTok{))}
\NormalTok{\}}
\NormalTok{gumbel <-}\StringTok{ }\ControlFlowTok{function}\NormalTok{(x, beta, gamma)\{ }
\NormalTok{  (}\DecValTok{1} \OperatorTok{/}\StringTok{ }\NormalTok{beta) }\OperatorTok{*}\KeywordTok{exp}\NormalTok{(}\OperatorTok{-}\KeywordTok{exp}\NormalTok{(}\OperatorTok{-}\NormalTok{(x}\OperatorTok{-}\NormalTok{gamma)}\OperatorTok{/}\NormalTok{beta) }\OperatorTok{-}\StringTok{ }\NormalTok{(x}\OperatorTok{-}\NormalTok{gamma)}\OperatorTok{/}\NormalTok{beta)}
\NormalTok{\}}
\NormalTok{logistic <-}\StringTok{ }\ControlFlowTok{function}\NormalTok{(x, beta, gamma)\{}
\NormalTok{  ( (}\DecValTok{1}\OperatorTok{/}\NormalTok{beta)}\OperatorTok{*}\KeywordTok{exp}\NormalTok{(}\OperatorTok{-}\NormalTok{(x}\OperatorTok{-}\NormalTok{gamma)}\OperatorTok{/}\NormalTok{beta) ) }\OperatorTok{/}\StringTok{ }\NormalTok{((}\DecValTok{1} \OperatorTok{+}\StringTok{ }\KeywordTok{exp}\NormalTok{(}\OperatorTok{-}\NormalTok{(x)))}\OperatorTok{^}\DecValTok{2}\NormalTok{)}
\NormalTok{\}}
\NormalTok{pareto <-}\StringTok{ }\ControlFlowTok{function}\NormalTok{(x, c, alpha)\{}
\NormalTok{  alpha}\OperatorTok{*}\NormalTok{((c}\OperatorTok{^}\NormalTok{alpha))}\OperatorTok{/}\NormalTok{(x}\OperatorTok{^}\NormalTok{(alpha}\OperatorTok{+}\DecValTok{1}\NormalTok{))}
\NormalTok{\}}
\end{Highlighting}
\end{Shaded}

\hypertarget{cauchy}{%
\paragraph{Cauchy}\label{cauchy}}

Para generar la muestra aleatoria de la distribución Cauchy podemos usar
el teorema de la transformada inversa y obtenemos la siguiente función:

\begin{Shaded}
\begin{Highlighting}[]
\NormalTok{rcauchy_ <-}\StringTok{ }\ControlFlowTok{function}\NormalTok{(n, beta, gamma)\{}
\NormalTok{  u <-}\StringTok{ }\KeywordTok{runif}\NormalTok{(n)}
\NormalTok{  cauchy_sample <-}\StringTok{ }\KeywordTok{tan}\NormalTok{(pi}\OperatorTok{*}\NormalTok{(u}\DecValTok{-1}\OperatorTok{/}\DecValTok{2}\NormalTok{))}\OperatorTok{*}\NormalTok{beta }\OperatorTok{+}\StringTok{ }\NormalTok{gamma}
  \KeywordTok{return}\NormalTok{ (cauchy_sample)}
\NormalTok{\}}
\end{Highlighting}
\end{Shaded}

Probando el método

\begin{Shaded}
\begin{Highlighting}[]
\NormalTok{beta <-}\StringTok{ }\DecValTok{1}
\NormalTok{gamma <-}\StringTok{ }\DecValTok{0}
\NormalTok{n <-}\StringTok{ }\DecValTok{5000}
\NormalTok{cauchy_sample =}\StringTok{ }\KeywordTok{rcauchy_}\NormalTok{(n, beta, gamma)}
\KeywordTok{hist}\NormalTok{(cauchy_sample, }\DataTypeTok{xlim =}\KeywordTok{c}\NormalTok{(}\OperatorTok{-}\DecValTok{10}\NormalTok{,}\DecValTok{10}\NormalTok{), }\DataTypeTok{breaks =} \KeywordTok{c}\NormalTok{(}\KeywordTok{min}\NormalTok{(cauchy_sample),}\KeywordTok{seq}\NormalTok{(}\DataTypeTok{from=}\OperatorTok{-}\DecValTok{10}\NormalTok{,}\DataTypeTok{to=}\DecValTok{10}\NormalTok{,}\DataTypeTok{by=}\NormalTok{.}\DecValTok{25}\NormalTok{), }\KeywordTok{max}\NormalTok{(cauchy_sample)), }\DataTypeTok{prob =}\NormalTok{ T)}
\KeywordTok{curve}\NormalTok{(}\KeywordTok{cauchy}\NormalTok{(x,beta,gamma),}\DataTypeTok{from=}\OperatorTok{-}\DecValTok{25}\NormalTok{,}\DataTypeTok{to=}\DecValTok{25}\NormalTok{,}\DataTypeTok{col=}\StringTok{"blue"}\NormalTok{, }\DataTypeTok{add =}\NormalTok{ T)}
\end{Highlighting}
\end{Shaded}

\includegraphics{sim-t3-sol_files/figure-latex/unnamed-chunk-10-1.pdf}

La muestra generada efectivamente sigue una distribución Cauchy.

Verificando empíricamente la ley fuerte de los grandes números:

\begin{Shaded}
\begin{Highlighting}[]
\NormalTok{n_values =}\StringTok{ }\KeywordTok{seq}\NormalTok{(}\DataTypeTok{from=}\DecValTok{50}\NormalTok{, }\DataTypeTok{to=}\DecValTok{5000}\NormalTok{, }\DataTypeTok{by=}\DecValTok{50}\NormalTok{)}
\NormalTok{x_barras =}\StringTok{ }\KeywordTok{rep}\NormalTok{(}\DecValTok{0}\NormalTok{,}\KeywordTok{length}\NormalTok{(n_values))}
\ControlFlowTok{for}\NormalTok{ (n }\ControlFlowTok{in}\NormalTok{ n_values)\{}
\NormalTok{  x_barras[n}\OperatorTok{/}\DecValTok{50}\NormalTok{] =}\StringTok{ }\KeywordTok{sum}\NormalTok{(}\KeywordTok{rcauchy_}\NormalTok{(n, beta, gamma))}\OperatorTok{/}\NormalTok{n}
\NormalTok{\}}
\KeywordTok{plot}\NormalTok{(x_barras)}
\end{Highlighting}
\end{Shaded}

\includegraphics{sim-t3-sol_files/figure-latex/unnamed-chunk-11-1.pdf}
Los valores oscilan alrededor del cero pero hay algunos valores que
difieren mucho, esto es normal puesto que la distribución Cauchy no
tiene media.

\hypertarget{gumbel}{%
\paragraph{Gumbel}\label{gumbel}}

Dado que la función de distribución es
\(e^{-e^{-\frac{x-\gamma}{\beta}}}\), invirtiendo tendremos
\(X = -\beta \log(-\log(u)) + \gamma\), usando el teorema de la
transformación inversa:

\begin{Shaded}
\begin{Highlighting}[]
\NormalTok{rgumbel_ <-}\StringTok{ }\ControlFlowTok{function}\NormalTok{(n,beta, gamma)\{}
\NormalTok{  u <-}\StringTok{ }\KeywordTok{runif}\NormalTok{(n)}
\NormalTok{  gumbel_sample <-}\StringTok{ }\OperatorTok{-}\NormalTok{beta}\OperatorTok{*}\KeywordTok{log}\NormalTok{(}\OperatorTok{-}\KeywordTok{log}\NormalTok{(u)) }\OperatorTok{+}\StringTok{ }\NormalTok{gamma}
  \KeywordTok{return}\NormalTok{ (gumbel_sample)}
\NormalTok{\}}
\end{Highlighting}
\end{Shaded}

Probando el método

\begin{Shaded}
\begin{Highlighting}[]
\NormalTok{gumbel_sample =}\StringTok{ }\KeywordTok{rgumbel_}\NormalTok{(}\DecValTok{5000}\NormalTok{,beta,gamma)}
\KeywordTok{hist}\NormalTok{(gumbel_sample, }\DataTypeTok{breaks =} \DecValTok{50}\NormalTok{, }\DataTypeTok{prob =}\NormalTok{ T)}
\KeywordTok{curve}\NormalTok{(}\KeywordTok{gumbel}\NormalTok{(x,beta,gamma),}\DataTypeTok{from=}\OperatorTok{-}\DecValTok{10}\NormalTok{,}\DataTypeTok{to=}\DecValTok{10}\NormalTok{,}\DataTypeTok{col=}\StringTok{"red"}\NormalTok{, }\DataTypeTok{add=}\NormalTok{T)}
\end{Highlighting}
\end{Shaded}

\includegraphics{sim-t3-sol_files/figure-latex/unnamed-chunk-13-1.pdf}
La muestra generada efectivamente sigue una distribución Gumbel.

Verificando empíricamente la ley fuerte de los grandes números:

\begin{Shaded}
\begin{Highlighting}[]
\NormalTok{x_barras =}\StringTok{ }\KeywordTok{rep}\NormalTok{(}\DecValTok{0}\NormalTok{,}\KeywordTok{length}\NormalTok{(n_values))}
\ControlFlowTok{for}\NormalTok{ (n }\ControlFlowTok{in}\NormalTok{ n_values)\{}
\NormalTok{  x_barras[n}\OperatorTok{/}\DecValTok{50}\NormalTok{] =}\StringTok{ }\KeywordTok{sum}\NormalTok{(}\KeywordTok{rgumbel_}\NormalTok{(n, beta, gamma))}\OperatorTok{/}\NormalTok{n}
\NormalTok{\}}
\KeywordTok{plot}\NormalTok{(x_barras)}
\KeywordTok{abline}\NormalTok{(}\DataTypeTok{h=}\NormalTok{(gamma }\OperatorTok{+}\StringTok{ }\NormalTok{beta}\OperatorTok{*}\FloatTok{0.5772}\NormalTok{), }\DataTypeTok{col=}\StringTok{'red'}\NormalTok{)}
\end{Highlighting}
\end{Shaded}

\includegraphics{sim-t3-sol_files/figure-latex/unnamed-chunk-14-1.pdf}
La media teórica es \(\gamma + \beta*c\) donde \(c\) es la constante de
Euler--Mascheroni, en la gráfica notamos que, efectivamente, se cumple
empíricamente la Ley fuerte de los grandes números.

\hypertarget{logistica}{%
\paragraph{Logística}\label{logistica}}

Dado que la función de distribución es
\(\frac{1}{1+e^{-(x-\gamma)/\beta}}\), invirtiendo tendremos
\(X = -\beta \log(\frac{1}{u} -1) + \gamma\), usando el teorema de la
transformación inversa:

\begin{Shaded}
\begin{Highlighting}[]
\NormalTok{rlogistic_ <-}\StringTok{ }\ControlFlowTok{function}\NormalTok{(n,beta,gamma)\{}
\NormalTok{  u <-}\StringTok{ }\KeywordTok{runif}\NormalTok{(n)}
\NormalTok{  logistic_sample <-}\StringTok{ }\OperatorTok{-}\NormalTok{beta}\OperatorTok{*}\KeywordTok{log}\NormalTok{(}\DecValTok{1}\OperatorTok{/}\NormalTok{u }\DecValTok{-1}\NormalTok{) }\OperatorTok{+}\StringTok{ }\NormalTok{gamma}
  \KeywordTok{return}\NormalTok{ (logistic_sample)}
\NormalTok{\}}
\end{Highlighting}
\end{Shaded}

Probando el método

\begin{Shaded}
\begin{Highlighting}[]
\NormalTok{logistic_sample =}\StringTok{ }\KeywordTok{rlogistic_}\NormalTok{(}\DecValTok{5000}\NormalTok{,beta,gamma)}
\KeywordTok{hist}\NormalTok{(logistic_sample, }\DataTypeTok{breaks =} \DecValTok{50}\NormalTok{, }\DataTypeTok{prob =}\NormalTok{ T)}
\KeywordTok{curve}\NormalTok{(}\KeywordTok{logistic}\NormalTok{(x,beta,gamma),}\DataTypeTok{from=}\OperatorTok{-}\DecValTok{10}\NormalTok{,}\DataTypeTok{to=}\DecValTok{10}\NormalTok{,}\DataTypeTok{col=}\StringTok{"red"}\NormalTok{, }\DataTypeTok{add=}\NormalTok{T)}
\end{Highlighting}
\end{Shaded}

\includegraphics{sim-t3-sol_files/figure-latex/unnamed-chunk-16-1.pdf}
La muestra generada efectivamente sigue una distribución Logística

Verificando empíricamente la ley fuerte de los grandes números:

\begin{Shaded}
\begin{Highlighting}[]
\NormalTok{x_barras =}\StringTok{ }\KeywordTok{rep}\NormalTok{(}\DecValTok{0}\NormalTok{,}\KeywordTok{length}\NormalTok{(n_values))}
\ControlFlowTok{for}\NormalTok{ (n }\ControlFlowTok{in}\NormalTok{ n_values)\{}
\NormalTok{  x_barras[n}\OperatorTok{/}\DecValTok{50}\NormalTok{] =}\StringTok{ }\KeywordTok{sum}\NormalTok{(}\KeywordTok{rlogistic_}\NormalTok{(n, beta, gamma))}\OperatorTok{/}\NormalTok{n}
\NormalTok{\}}
\KeywordTok{plot}\NormalTok{(x_barras)}
\KeywordTok{abline}\NormalTok{(}\DataTypeTok{h=}\NormalTok{gamma,}\DataTypeTok{col=}\StringTok{"red"}\NormalTok{)}
\end{Highlighting}
\end{Shaded}

\includegraphics{sim-t3-sol_files/figure-latex/unnamed-chunk-17-1.pdf}
La media teórica es \(\gamma\). En la gráfica notamos que,
efectivamente, se cumple empíricamente la Ley fuerte de los grandes
números.

\hypertarget{pareto}{%
\paragraph{Pareto}\label{pareto}}

La inversa de la función de distribución está dada por
\(X = \frac{c}{u^{1/\alpha}}\), usando el teorema de la transformada
inversa:

\begin{Shaded}
\begin{Highlighting}[]
\NormalTok{rpareto_ <-}\StringTok{ }\ControlFlowTok{function}\NormalTok{(n,c,alpha)\{}
\NormalTok{  u <-}\StringTok{ }\KeywordTok{runif}\NormalTok{(n)}
\NormalTok{  pareto_sample <-}\StringTok{ }\NormalTok{c }\OperatorTok{/}\StringTok{ }\NormalTok{(u}\OperatorTok{^}\NormalTok{(}\DecValTok{1}\OperatorTok{/}\NormalTok{alpha))}
\NormalTok{\}}
\end{Highlighting}
\end{Shaded}

Probando el método

\begin{Shaded}
\begin{Highlighting}[]
\NormalTok{c <-}\StringTok{ }\DecValTok{1}
\NormalTok{alpha <-}\StringTok{ }\DecValTok{2}
\NormalTok{pareto_sample =}\StringTok{ }\KeywordTok{rpareto_}\NormalTok{(}\DecValTok{5000}\NormalTok{,c,alpha)}
\KeywordTok{hist}\NormalTok{(pareto_sample, }\DataTypeTok{xlim=}\KeywordTok{c}\NormalTok{(}\DecValTok{1}\NormalTok{,}\DecValTok{5}\NormalTok{), }\DataTypeTok{breaks =} \KeywordTok{c}\NormalTok{(}\KeywordTok{seq}\NormalTok{(}\DataTypeTok{from=}\OperatorTok{-}\DecValTok{10}\NormalTok{,}\DataTypeTok{to=}\DecValTok{10}\NormalTok{,}\DataTypeTok{by=}\NormalTok{.}\DecValTok{25}\NormalTok{),}\KeywordTok{max}\NormalTok{(pareto_sample)), }\DataTypeTok{prob =}\NormalTok{ T)}
\KeywordTok{curve}\NormalTok{(}\KeywordTok{pareto}\NormalTok{(x,c,alpha),}\DataTypeTok{from=}\DecValTok{1}\NormalTok{,}\DataTypeTok{to=}\DecValTok{5}\NormalTok{,}\DataTypeTok{col=}\StringTok{"blue"}\NormalTok{, }\DataTypeTok{add =}\NormalTok{ T)}
\end{Highlighting}
\end{Shaded}

\includegraphics{sim-t3-sol_files/figure-latex/unnamed-chunk-19-1.pdf}
La muestra generada efectivamente sigue una distribución Pareto.

Verificando empíricamente la ley fuerte de los grandes números:

\begin{Shaded}
\begin{Highlighting}[]
\NormalTok{x_barras =}\StringTok{ }\KeywordTok{rep}\NormalTok{(}\DecValTok{0}\NormalTok{,}\KeywordTok{length}\NormalTok{(n_values))}
\ControlFlowTok{for}\NormalTok{ (n }\ControlFlowTok{in}\NormalTok{ n_values)\{}
\NormalTok{  x_barras[n}\OperatorTok{/}\DecValTok{50}\NormalTok{] =}\StringTok{ }\KeywordTok{sum}\NormalTok{(}\KeywordTok{rpareto_}\NormalTok{(n, c, alpha))}\OperatorTok{/}\NormalTok{n}
\NormalTok{\}}
\KeywordTok{plot}\NormalTok{(x_barras)}
\KeywordTok{abline}\NormalTok{(}\DataTypeTok{h=}\DecValTok{2}\NormalTok{,}\DataTypeTok{col=}\StringTok{"red"}\NormalTok{)}
\end{Highlighting}
\end{Shaded}

\includegraphics{sim-t3-sol_files/figure-latex/unnamed-chunk-20-1.pdf}
La media teórica es \$\frac{c \ \alpha}{\alpha -1} = 2 \$. En la gráfica
notamos que, efectivamente, se cumple empíricamente la Ley fuerte de los
grandes números.

\hypertarget{pregunta-3-1}{%
\subsubsection{Pregunta 3}\label{pregunta-3-1}}

Grafique las densidades indicadas, y dé algoritmos de transformación
inversa, composición y aceptación-rechazo para cada una de ellas.

\begin{enumerate}
\def\labelenumi{\alph{enumi}.}
\tightlist
\item
  \[
  f(x) = \frac{3x^2}{2} \mathcal{I}_{[-1,1]}(x)
  \]
\end{enumerate}

\begin{Shaded}
\begin{Highlighting}[]
\KeywordTok{ggplot}\NormalTok{(}\KeywordTok{data.frame}\NormalTok{(}\DataTypeTok{x=}\KeywordTok{c}\NormalTok{(}\OperatorTok{-}\FloatTok{1.2}\NormalTok{,}\FloatTok{1.2}\NormalTok{)), }\KeywordTok{aes}\NormalTok{(x)) }\OperatorTok{+}
\StringTok{  }\KeywordTok{stat_function}\NormalTok{(}\DataTypeTok{fun =} \ControlFlowTok{function}\NormalTok{(x) }\KeywordTok{ifelse}\NormalTok{(x}\OperatorTok{>=}\StringTok{ }\DecValTok{-1} \OperatorTok{&}\StringTok{ }\NormalTok{x }\OperatorTok{<=}\StringTok{ }\DecValTok{1}\NormalTok{, }\DecValTok{3}\OperatorTok{/}\DecValTok{2}\OperatorTok{*}\NormalTok{x}\OperatorTok{^}\DecValTok{2}\NormalTok{, }\DecValTok{0}\NormalTok{)) }\OperatorTok{+}
\StringTok{  }\KeywordTok{labs}\NormalTok{(}\DataTypeTok{x=}\StringTok{'x'}\NormalTok{, }\DataTypeTok{y=}\StringTok{'f(x)'}\NormalTok{)}
\end{Highlighting}
\end{Shaded}

\includegraphics{sim-t3-sol_files/figure-latex/unnamed-chunk-21-1.pdf}

Y \(F(x) = \int_{-1}^x \frac{3t^2}{2} dt = \frac{1}{2}(x^3+1)\), por lo
que la función cuantil es \(F^{-1}(x) = \sqrt[3]{2x^3-1}\).

El método de aceptación-rechazo sería generar parejas \((u,v)\), donde
\(u \sim \textrm{Unif}(-1,1)\) y \(v \sim \textrm{Unif}(0,1.5)\), y nos
quedamos con aquellos puntos que satisfacen \(v \leq f(u)\). Notemos que
la tasa de aceptación es 1/3.

\begin{enumerate}
\def\labelenumi{\alph{enumi}.}
\setcounter{enumi}{1}
\tightlist
\item
  \[
  g(x) = 
  \begin{cases}
  0, x\leq 0 \\
  \frac{x}{a(1-a)}, 0 < x \leq a \\
  \frac{1}{1-a}, a < x \leq 1-a \\
  \frac{1-x}{a(1-a)}, 1-a < x \leq 1
  0, x  > 1
  \end{cases}
  \]
\end{enumerate}

\begin{Shaded}
\begin{Highlighting}[]
\NormalTok{g <-}\StringTok{ }\ControlFlowTok{function}\NormalTok{(a, x)\{}
  \KeywordTok{ifelse}\NormalTok{(x}\OperatorTok{<=}\DecValTok{0}\NormalTok{, }\DecValTok{0}\NormalTok{,}
         \KeywordTok{ifelse}\NormalTok{(x }\OperatorTok{<=}\StringTok{ }\NormalTok{a, x}\OperatorTok{/}\NormalTok{(a}\OperatorTok{*}\NormalTok{(}\DecValTok{1}\OperatorTok{-}\NormalTok{a)), }
                \KeywordTok{ifelse}\NormalTok{(x }\OperatorTok{<=}\StringTok{ }\DecValTok{1}\OperatorTok{-}\NormalTok{a, }\DecValTok{1}\OperatorTok{/}\NormalTok{(}\DecValTok{1}\OperatorTok{-}\NormalTok{a), }
                       \KeywordTok{ifelse}\NormalTok{(x}\OperatorTok{<=}\StringTok{ }\DecValTok{1}\NormalTok{, (}\DecValTok{1}\OperatorTok{-}\NormalTok{x)}\OperatorTok{/}\NormalTok{(a}\OperatorTok{*}\NormalTok{(}\DecValTok{1}\OperatorTok{-}\NormalTok{a)),}\DecValTok{0}\NormalTok{))))}
\NormalTok{\}}

\KeywordTok{ggplot}\NormalTok{(}\KeywordTok{data.frame}\NormalTok{(}\DataTypeTok{x=}\KeywordTok{c}\NormalTok{(}\DecValTok{0}\NormalTok{,}\DecValTok{1}\NormalTok{)), }\KeywordTok{aes}\NormalTok{(x)) }\OperatorTok{+}
\StringTok{  }\KeywordTok{stat_function}\NormalTok{(}\DataTypeTok{fun =} \ControlFlowTok{function}\NormalTok{(x) }\KeywordTok{g}\NormalTok{(}\DecValTok{1}\OperatorTok{/}\DecValTok{3}\NormalTok{, x)) }\OperatorTok{+}\StringTok{ }
\StringTok{  }\KeywordTok{labs}\NormalTok{(}\DataTypeTok{y=}\StringTok{'g(x)'}\NormalTok{, }\DataTypeTok{x=}\StringTok{'x'}\NormalTok{)}
\end{Highlighting}
\end{Shaded}

\includegraphics{sim-t3-sol_files/figure-latex/unnamed-chunk-22-1.pdf}

En este caso, la tasa de aceptación para el algoritmo de
aceptación-rechazo es 2/3, mejor que arriba.

Para transformación inversa, la función de distribución es \[
G(x) = 
\begin{cases}
0, x \leq 0 \\
\frac{a^2-x^2}{2(a-1)a}, 0 < x \leq a\\
\frac{a-x}{a-1}, a < x \leq 1-a \\
\frac{(x-1)^2-a^2}{2(a-1)a}, 1-a < x \leq 1 \\
0, x > 1
\end{cases}
\]

\hypertarget{pregunta-5}{%
\subsection{Pregunta 5}\label{pregunta-5}}

Considerando la transformación polar de Marsaglia para generar muestras
de normales estándar, muestren que la probabilidad de aceptación de
\(S = V12 + V22\) en el paso 2 es \(\pi/4\), y encuentren la
distribución del número de rechazos de S antes de que ocurra una
aceptación. ¿Cuál es el número esperado de ejecuciones del paso 1?

Se tienen \(V_1, V_2 \sim Unif(-1,1)\) y \(S=V_1^2 + V_2^2\) si
\(V_1^2 + V_2^2 > 1\). Por un lado, el area de aceptación es un circulo
de radio 1 y centrado en cero. Por otro lado el area total es el
recuadro \([-1,1] \times [-1,1]\), el área del círculo es \(\pi\) y el
area del recuadro es igual a dos por lo que la probabilidad de
aceptación es \(\frac{\pi}{4}\)

El número de rechazos de S antes de que ocurra una aceptación se
distribuye como una geométrica con parámetro \(\frac{\pi}{4}\).

Si \(X \sim Geom(\frac{\pi}{4})\), entonces eel número esperado de
ejecuciones del paso 1 es igual a la esperanza X,
\(E[X] = \frac{1}{\frac{\pi}{4}} = \frac{4}{pi}\)

Que tiene función cuantil más o menos fea por la inversión de cosas no
biyectivas, por lo que mejor podríamos usar el método de
aceptación-rechazo.

\hypertarget{pregunta-6}{%
\subsection{Pregunta 6}\label{pregunta-6}}

Obtenga 1000 números de la distribución \[
p(x) = \frac{2x}{k(k+1)}, x\in\{1, \cdots, k\}
\] para \(k = 100\).

\begin{Shaded}
\begin{Highlighting}[]
\NormalTok{construir_tabla <-}\StringTok{ }\ControlFlowTok{function}\NormalTok{(k)\{}
\NormalTok{  x <-}\StringTok{ }\DecValTok{1}\OperatorTok{:}\NormalTok{k}
  \DecValTok{2}\OperatorTok{*}\NormalTok{x}\OperatorTok{/}\NormalTok{(k}\OperatorTok{*}\NormalTok{k}\OperatorTok{+}\DecValTok{1}\NormalTok{)}
\NormalTok{\}}

\NormalTok{rp <-}\StringTok{ }\ControlFlowTok{function}\NormalTok{(n, k)\{}
\NormalTok{  pk <-}\StringTok{ }\DecValTok{2}\OperatorTok{*}\DecValTok{1}\OperatorTok{:}\NormalTok{k}\OperatorTok{/}\NormalTok{(k}\OperatorTok{*}\NormalTok{(k}\OperatorTok{+}\DecValTok{1}\NormalTok{))}
  \KeywordTok{sample}\NormalTok{(}\DecValTok{1}\OperatorTok{:}\NormalTok{k, n, }\DataTypeTok{replace =}\NormalTok{ T, }\DataTypeTok{prob=}\NormalTok{pk)}
\NormalTok{\}}

\KeywordTok{rp}\NormalTok{(}\DecValTok{1000}\NormalTok{,}\DecValTok{10}\NormalTok{)}
\end{Highlighting}
\end{Shaded}

\begin{verbatim}
##    [1]  9  8 10  6  2 10  8  6 10  3  9  2  8 10  7  2 10 10 10  7  1 10 10
##   [24]  5  5  5  7  4  7  7  5  3  7  9  9  8  9 10 10  9  9  7  7  7  6  6
##   [47]  6 10  7  5  5  8  6  9  6  2 10  2  8  7  6  5  6  9 10  4 10  5 10
##   [70]  8 10  4  6  8  9  4  7  2  1 10 10  9  5  3  3  8  6  6  7  1  9  7
##   [93]  7 10  8  3  2  7  5 10  4  5  5  3  7  6  3  4 10  6  2  9  5  8  9
##  [116] 10  5  7  6  8  9  9  5  3  4  8  7 10  4  3  5 10  7  8  5  7  7  9
##  [139]  3 10 10  8  5  9 10  4 10  9  7 10  5  1  6  6  9  7 10  7 10  4  4
##  [162]  7  9  2  7  5  8 10  7 10  8  5  5  6  2 10  8  8  6  6  9  6 10 10
##  [185]  7  4  5  4  6 10  9  8  5 10 10  7  9  7 10  6  4  7  8  9  7  3  7
##  [208]  2 10  4  6  5  6  4  3  7 10  9  7  4  1  2  3  9  9  7  5  9  5  8
##  [231] 10 10  2  5  7 10  8  2  6 10  9  8 10 10  3  7 10  9  8  3  9  8  8
##  [254]  7 10  4  9  2  8  6  8  6  4  8  9  5  7  6  4 10  6  7  9 10  8  6
##  [277] 10 10  4 10  9  5  3  7  5  6  7 10  9  3 10  5 10  5  8 10 10  9  4
##  [300]  2 10  7  7  3 10  7  9  3  5  9  8  3  6  7  3 10  6 10  9  8  8  5
##  [323]  9  8  7 10  9  1  9  4 10  3  4  1  8  9  6  8  6 10  3  5  7  7  9
##  [346] 10  7  9  9 10  7 10  5  8  9 10  6 10  2 10  9  4  9  4  7 10 10 10
##  [369] 10  5  7 10  3  6  5  5  1  9  9 10  7  5  7  4  8  7  8  8  8  6 10
##  [392]  9  9 10  7 10 10  6 10  6  4  9  7  7  9  5 10  4  9  6  4  6 10  8
##  [415]  8  8  9  9  6 10  9  6  5  6  7  9  5  9  7 10  9  8  7  8  4  2 10
##  [438]  3 10  7 10  6  9  3 10  5  7  7  4  7  8  7  6  3 10  8 10  7  3  8
##  [461]  6 10  5  6  9  9 10 10  6  3  7  4  6  6 10  6  2  9  8  7  8  7  8
##  [484]  8  8  7  6  6  7  5 10  9 10  9 10 10  3 10  6  2  7  9  5  4  9  8
##  [507]  9  7  7  6  7  8  9  6  4  6  8  8  9  1  3 10  1  8 10  8  9 10  2
##  [530] 10  8  4  9  4  9  9  5  7  8  4  8 10  7 10  8 10  5  8  5  3  8  9
##  [553] 10  4  2  9  7  9  6  8 10  7  7  4 10  5  5 10  6  3  9  2  8 10  8
##  [576]  7 10  7  8 10  8  6  9  7  4 10 10 10  5 10  9  5  8 10  9 10  6  4
##  [599]  7  6  4  3  3  9  8  9  3  9  9 10  9  7  8  7  6  3  7  8  9 10  8
##  [622] 10  6 10  8  1  8  9  6  4 10  7  9  6  5  5 10  8  9  9 10  8 10  8
##  [645] 10  1  4  5  8 10  7  7  8  3  9  9  4  6  8  3  5  7 10  9 10  3  7
##  [668]  8  2  9 10  8 10  8  4  6  8  9  8  9 10  9  6  4  8  3  9  4  5  8
##  [691] 10  4  3  3  6  3  6  6  4  8  6  4  5  9  7  8  8  8  9  9  7 10 10
##  [714]  3  2  9  3 10  6  6  7  3  5  8  2  8  8  3  1  2  3 10  8  6  8  5
##  [737]  6  9  9  5 10  4  2 10 10  2 10  1  9  4  9  3  9  8  5  5  8  8  7
##  [760]  8 10  7  2  9  9  7  7  8  8 10 10 10  9  7  8  2 10  7  8  8  7  6
##  [783]  9  4  9  9 10 10 10  1  8 10  9 10  2  4  6  7  6 10  8 10  9 10  7
##  [806]  3  2  2 10  6  3  8  9  7  8  7 10  2 10  2  5  7  8  3  5  9  6 10
##  [829]  4  7  8  7  9 10  6  9 10  4  8  6  4  5  9  7  9  7  2  8  8 10  5
##  [852]  4  9 10  5 10  8 10  6 10  9  8  5  5  6  9  7  5 10  7  7  9  8  8
##  [875]  5  7  9  9  9 10  9  6  7  7 10  1  9  8  3  2  8  1  6  8  6  5  1
##  [898]  9  8  8 10  8  7 10  3 10  5  5  9 10  4  7  5  3  9  9 10  7  9  4
##  [921] 10 10  8  9  7  9  5 10  3  5  5  5  7 10  5 10 10 10  8  8  5  8  9
##  [944]  3  4  7  9  7  3  8 10  6  6  9  4  6  5  2  7 10  6  6  6  2  6 10
##  [967]  7  6  6  7  8  1  8  9 10  9  9  8  8  8 10  9  7  9  9  6  7  7  9
##  [990] 10  7  6  8  9  5  5  6  2 10  9
\end{verbatim}

\hypertarget{pregunta-7}{%
\subsection{Pregunta 7}\label{pregunta-7}}

Desarrollen un algoritmo para generar una variable aleatoria binomial,
usando la técnica de convolución (Hint: ¿cuál es la relación entre
binomiales y Bernoullis?) Generar una muestra de 100,000 números. ¿Qué
método es más eficiente, el de con- voluciones o la función rbinom en R?

Podemos usar el método de la convolución para generar la muestra
aleatoria de una Binomial usando el hecho de que si
\(X_i \sim Ber(p) \quad i\in\{1,...,n\}\) entonces
\(\sum_{i=1}^n Xi \sim Bin(n,p)\).

La siguiente función utiliza dicha información para generar la muestra
aleatoria de tamaño n de una variable aleatoria binomial de parámetros
n,p:

\begin{Shaded}
\begin{Highlighting}[]
\NormalTok{rbinom_ <-}\ControlFlowTok{function}\NormalTok{(n,size,p)\{}
\NormalTok{  binomial_sample <-}\StringTok{ }\KeywordTok{rep}\NormalTok{(}\DecValTok{0}\NormalTok{,n)}
  \ControlFlowTok{for}\NormalTok{ (i }\ControlFlowTok{in} \DecValTok{1}\OperatorTok{:}\NormalTok{n)\{}
\NormalTok{    binomial_sample[i] <-}\StringTok{ }\KeywordTok{sum}\NormalTok{(}\KeywordTok{sample}\NormalTok{(}\DecValTok{0}\OperatorTok{:}\DecValTok{1}\NormalTok{, }\DataTypeTok{size=}\NormalTok{size,}\DataTypeTok{replace=}\NormalTok{T, }\DataTypeTok{prob=}\KeywordTok{c}\NormalTok{(}\DecValTok{1}\OperatorTok{-}\NormalTok{p,p)))}
\NormalTok{  \}}
  \KeywordTok{return}\NormalTok{(binomial_sample)}
\NormalTok{\}}
\end{Highlighting}
\end{Shaded}

Finalmente, comparemos los tiempos de ejecución para ver cual es más
eficiente:

\begin{Shaded}
\begin{Highlighting}[]
\KeywordTok{system.time}\NormalTok{(}\KeywordTok{rbinom_}\NormalTok{(}\DecValTok{1000}\NormalTok{,}\DecValTok{10}\NormalTok{,.}\DecValTok{5}\NormalTok{))}
\end{Highlighting}
\end{Shaded}

\begin{verbatim}
##    user  system elapsed 
##   0.010   0.000   0.011
\end{verbatim}

\begin{Shaded}
\begin{Highlighting}[]
\KeywordTok{system.time}\NormalTok{(}\KeywordTok{rbinom}\NormalTok{(}\DecValTok{1000}\NormalTok{,}\DecValTok{10}\NormalTok{,.}\DecValTok{5}\NormalTok{))}
\end{Highlighting}
\end{Shaded}

\begin{verbatim}
##    user  system elapsed 
##       0       0       0
\end{verbatim}

Como era de esperarse, la función de R es más eficiente. El algoritmo
utilizado por R es llamado el algoritmo BTPEC para samplear normales
(Kachitvichyanukul, V. and Schmeiser, B. W. (1988). Binomial random
variate generation.)

\hypertarget{pregunta-8}{%
\subsection{Pregunta 8}\label{pregunta-8}}

Para un proceso Poisson no homogéneo con función de intensidad dada por

\[
\lambda(t)= 
\begin{cases}
5, t\in(1,2],(3,4], \cdots \\
3, t \in (0,1], (2,3], \cdots 
\end{cases}
\]

Grafique un ejemplo del proceso, considerando en intervalo de tiempo
\([0,100]\)

\begin{Shaded}
\begin{Highlighting}[]
\CommentTok{# Generar primeros N eventos}
\NormalTok{poisson_nh <-}\StringTok{ }\ControlFlowTok{function}\NormalTok{(lambda, cota, }\DataTypeTok{N=}\DecValTok{0}\NormalTok{, }\DataTypeTok{Tmax=}\OtherTok{Inf}\NormalTok{, }\DataTypeTok{by_time=}\NormalTok{T)\{}
\NormalTok{  t_inter <-}\StringTok{ }\DecValTok{0}
\NormalTok{  t_actual <-}\StringTok{ }\DecValTok{0}
\NormalTok{  tiempo_evento <-}\StringTok{ }\KeywordTok{integer}\NormalTok{()}
\NormalTok{  n <-}\StringTok{ }\DecValTok{0}
  
  \ControlFlowTok{while}\NormalTok{((by_time }\OperatorTok{&}\StringTok{ }\NormalTok{t_actual}\OperatorTok{<}\NormalTok{Tmax) }\OperatorTok{|}\StringTok{ }\NormalTok{(}\OperatorTok{!}\NormalTok{by_time }\OperatorTok{&}\StringTok{ }\NormalTok{n}\OperatorTok{<}\NormalTok{N))\{}
\NormalTok{    t_inter <-}\StringTok{ }\KeywordTok{rexp}\NormalTok{(}\DecValTok{1}\NormalTok{, cota)}
\NormalTok{    t_actual <-}\StringTok{ }\NormalTok{t_actual }\OperatorTok{+}\StringTok{ }\NormalTok{t_inter}
    \ControlFlowTok{if}\NormalTok{(}\KeywordTok{runif}\NormalTok{(}\DecValTok{1}\NormalTok{) }\OperatorTok{<}\StringTok{ }\KeywordTok{lambda}\NormalTok{(t_actual)}\OperatorTok{/}\NormalTok{cota)\{}
\NormalTok{      n <-}\StringTok{ }\NormalTok{n}\OperatorTok{+}\DecValTok{1}
      \ControlFlowTok{if}\NormalTok{(t_actual}\OperatorTok{<}\NormalTok{Tmax)}
\NormalTok{        tiempo_evento <-}\StringTok{ }\KeywordTok{c}\NormalTok{(tiempo_evento, t_actual)}
\NormalTok{    \}}
\NormalTok{  \}}
  
\NormalTok{  tiempo_evento}
\NormalTok{\}}


\NormalTok{lambda <-}\StringTok{ }\ControlFlowTok{function}\NormalTok{(t)\{}
  \KeywordTok{ifelse}\NormalTok{(}\KeywordTok{floor}\NormalTok{(t)}\OperatorTok\DecValTok{2}\OperatorTok{==}\DecValTok{1}\NormalTok{, }\DecValTok{5}\NormalTok{, }\DecValTok{3}\NormalTok{)}
\NormalTok{\}}

\NormalTok{poisson_plot<-}\StringTok{ }\ControlFlowTok{function}\NormalTok{(t)\{}
  \KeywordTok{data_frame}\NormalTok{(}\DataTypeTok{ns=}\DecValTok{1}\OperatorTok{:}\KeywordTok{length}\NormalTok{(t),}
           \DataTypeTok{tiempo=}\NormalTok{t) }\OperatorTok
\StringTok{  }\KeywordTok{ggplot}\NormalTok{(}\KeywordTok{aes}\NormalTok{(}\DataTypeTok{x=}\NormalTok{t, }\DataTypeTok{y=}\NormalTok{ns)) }\OperatorTok{+}
\StringTok{  }\KeywordTok{geom_step}\NormalTok{() }\OperatorTok{+}
\StringTok{  }\KeywordTok{labs}\NormalTok{(}\DataTypeTok{x=}\StringTok{'t'}\NormalTok{, }\DataTypeTok{y=}\StringTok{'N(t)'}\NormalTok{)}
\NormalTok{\}}

\KeywordTok{poisson_nh}\NormalTok{(lambda, }\DecValTok{5}\NormalTok{, }\DataTypeTok{Tmax=}\DecValTok{100}\NormalTok{) }\OperatorTok
\StringTok{  }\KeywordTok{poisson_plot}\NormalTok{()}
\end{Highlighting}
\end{Shaded}

\includegraphics{sim-t3-sol_files/figure-latex/unnamed-chunk-26-1.pdf}

Grafique el proceso hasta obtener 100 eventos

\begin{Shaded}
\begin{Highlighting}[]
\KeywordTok{poisson_nh}\NormalTok{(lambda, }\DecValTok{5}\NormalTok{, }\DataTypeTok{N=}\DecValTok{100}\NormalTok{, }\DataTypeTok{by_time=}\NormalTok{F) }\OperatorTok
\StringTok{  }\KeywordTok{poisson_plot}\NormalTok{()}
\end{Highlighting}
\end{Shaded}

\includegraphics{sim-t3-sol_files/figure-latex/unnamed-chunk-27-1.pdf}

Estime la probabilidad de que el número de eventos observados en el
periodo de tiempo (1.25,3{]} sea mayor a dos.

\begin{Shaded}
\begin{Highlighting}[]
\NormalTok{count_obs <-}\StringTok{ }\ControlFlowTok{function}\NormalTok{(t, lb)\{}

  \ControlFlowTok{if}\NormalTok{(}\KeywordTok{length}\NormalTok{(}\KeywordTok{which}\NormalTok{(t}\OperatorTok{<}\NormalTok{lb))}\OperatorTok{==}\DecValTok{0}\NormalTok{)}
    \DecValTok{0}
  \ControlFlowTok{else}
    \KeywordTok{length}\NormalTok{(t) }\OperatorTok{-}\StringTok{ }\KeywordTok{max}\NormalTok{(}\KeywordTok{which}\NormalTok{(t}\OperatorTok{<}\NormalTok{lb))}
\NormalTok{\}}

\KeywordTok{replicate}\NormalTok{(}\DecValTok{10000}\NormalTok{, }\KeywordTok{poisson_nh}\NormalTok{(lambda, }\DecValTok{5}\NormalTok{, }\DataTypeTok{Tmax=}\DecValTok{3}\NormalTok{)) }\OperatorTok
\StringTok{  }\KeywordTok{map_dbl}\NormalTok{(count_obs, }\FloatTok{1.25}\NormalTok{) }\OperatorTok
\StringTok{  }\KeywordTok{map_lgl}\NormalTok{(}\ControlFlowTok{function}\NormalTok{(x) x }\OperatorTok{>}\StringTok{ }\DecValTok{2}\NormalTok{) }\OperatorTok
\StringTok{  }\KeywordTok{mean}\NormalTok{() }\OperatorTok
\StringTok{  }\KeywordTok{print}\NormalTok{()}
\end{Highlighting}
\end{Shaded}

\begin{verbatim}
## [1] 0.9508
\end{verbatim}

\hypertarget{pregunta-9}{%
\subsection{Pregunta 9}\label{pregunta-9}}

Simular un proceso Poisson no homogéneo con función de intensidad dada
por \(\lambda(t) = sen(t)\).

La siguiente función genera el proceso:

\begin{Shaded}
\begin{Highlighting}[]
\NormalTok{non_homogeneous_poisson_process <-}\StringTok{ }\ControlFlowTok{function}\NormalTok{(t)\{}
  \CommentTok{# lambda = 1 acota la función de intensidad del proceso}
\NormalTok{  lambda <-}\StringTok{ }\DecValTok{1}

\NormalTok{  s =}\StringTok{ }\KeywordTok{c}\NormalTok{(}\KeywordTok{rexp}\NormalTok{(}\DecValTok{1}\NormalTok{,}\DecValTok{1}\NormalTok{))}
  \ControlFlowTok{while}\NormalTok{(}\KeywordTok{tail}\NormalTok{(s,}\DecValTok{1}\NormalTok{) }\OperatorTok{<}\StringTok{ }\NormalTok{t)\{}
\NormalTok{    s =}\StringTok{ }\KeywordTok{c}\NormalTok{(s, }\KeywordTok{tail}\NormalTok{(s,}\DecValTok{1}\NormalTok{)}\OperatorTok{+}\KeywordTok{rexp}\NormalTok{(}\DecValTok{1}\NormalTok{,}\DecValTok{1}\NormalTok{))}
\NormalTok{  \}}
\NormalTok{  s =}\StringTok{ }\NormalTok{s[}\OperatorTok{-}\KeywordTok{length}\NormalTok{(s)]}
  
\NormalTok{  u <-}\StringTok{ }\KeywordTok{runif}\NormalTok{(}\KeywordTok{length}\NormalTok{(s))}
\NormalTok{  ss <-}\StringTok{ }\NormalTok{s[u }\OperatorTok{<=}\StringTok{ }\KeywordTok{abs}\NormalTok{(}\KeywordTok{sin}\NormalTok{(s)}\OperatorTok{/}\NormalTok{lambda)]}
\NormalTok{  Ns <-}\StringTok{ }\DecValTok{1}\OperatorTok{:}\KeywordTok{length}\NormalTok{(ss)}
  
  \KeywordTok{return}\NormalTok{(}\KeywordTok{list}\NormalTok{(}\DataTypeTok{intentados=}\NormalTok{s, }\DataTypeTok{aceptados =}\NormalTok{ ss, }\DataTypeTok{cuenta=}\NormalTok{ Ns))}
\NormalTok{\}}
\end{Highlighting}
\end{Shaded}

Simulando el proceso.

\begin{Shaded}
\begin{Highlighting}[]
\NormalTok{ans =}\StringTok{ }\KeywordTok{non_homogeneous_poisson_process}\NormalTok{(}\DecValTok{50}\NormalTok{)}
\KeywordTok{par}\NormalTok{(}\DataTypeTok{mfrow=}\KeywordTok{c}\NormalTok{(}\DecValTok{1}\NormalTok{,}\DecValTok{2}\NormalTok{), }\DataTypeTok{pty=}\StringTok{'s'}\NormalTok{)}
\KeywordTok{plot}\NormalTok{(ans}\OperatorTok{$}\NormalTok{aceptados, ans}\OperatorTok{$}\NormalTok{cuenta, }\DataTypeTok{type =} \StringTok{"s"}\NormalTok{, }\DataTypeTok{ylab =} \StringTok{"N(t)"}\NormalTok{,}\DataTypeTok{xlab=}\StringTok{'t'}\NormalTok{,}
     \DataTypeTok{main =} \StringTok{"Proceso Poisson no homogéneo"}\NormalTok{,}
     \DataTypeTok{sub =} \KeywordTok{expression}\NormalTok{(}\KeywordTok{lambda}\NormalTok{(t) }\OperatorTok{==}\StringTok{ }\KeywordTok{exp}\NormalTok{(}\KeywordTok{paste}\NormalTok{(}\StringTok{"sen"}\NormalTok{,}\StringTok{"(t)"}\NormalTok{)))) }

\KeywordTok{plot}\NormalTok{(ans}\OperatorTok{$}\NormalTok{intentados, }\KeywordTok{sin}\NormalTok{(ans}\OperatorTok{$}\NormalTok{intentados),}\DataTypeTok{col =} \StringTok{"red"}\NormalTok{, }\DataTypeTok{lwd =} \DecValTok{1}\NormalTok{ )}
\KeywordTok{points}\NormalTok{(ans}\OperatorTok{$}\NormalTok{aceptados, }\KeywordTok{sin}\NormalTok{(ans}\OperatorTok{$}\NormalTok{aceptados),}\DataTypeTok{col =} \StringTok{"blue"}\NormalTok{, }\DataTypeTok{lwd =} \DecValTok{1}\NormalTok{)}
\KeywordTok{curve}\NormalTok{(}\KeywordTok{sin}\NormalTok{(x), }\DataTypeTok{from=}\DecValTok{0}\NormalTok{, }\DataTypeTok{to=}\DecValTok{100}\NormalTok{, }\DataTypeTok{add=}\NormalTok{T)}
\KeywordTok{abline}\NormalTok{(}\DataTypeTok{h=}\OperatorTok{-}\NormalTok{.}\DecValTok{5}\NormalTok{)}
\KeywordTok{abline}\NormalTok{(}\DataTypeTok{h=}\NormalTok{.}\DecValTok{5}\NormalTok{)}
\end{Highlighting}
\end{Shaded}

\includegraphics{sim-t3-sol_files/figure-latex/unnamed-chunk-30-1.pdf}
La gráfica de la izquierda es el proceso simulado, la gráfica de la
derecha es la gráfica de sen(t), los puntos sobre la curva son aquellos
puntos de la forma \((t,sen(t))\), los puntos rojos fueron rechazados,
en caso contrario fueron aceptados en el proceso.

\hypertarget{pregunta-10}{%
\subsection{Pregunta 10}\label{pregunta-10}}

Una compañía de seguros tiene 1000 asegurados, cada uno de los cuales
presentará de manera independiente una reclamación en el siguiente mes
con probabilidad \(p=0.09245\). Suponiendo que las cantidades de los
reclamos hechos son variables aleatorias
\(\mathcal{N}(7000, 25000000)\), estime la probabilidad de que la suma
de los reclamos exceda \$500,000.

\begin{Shaded}
\begin{Highlighting}[]
\NormalTok{simular_mes <-}\StringTok{ }\ControlFlowTok{function}\NormalTok{(}\DataTypeTok{n_clientes=}\DecValTok{1000}\NormalTok{, }\DataTypeTok{p=}\FloatTok{0.09245}\NormalTok{, }\DataTypeTok{mu=}\DecValTok{7000}\NormalTok{, }\DataTypeTok{sigma=}\DecValTok{5000}\NormalTok{)\{}
\NormalTok{  reclamacion <-}\StringTok{ }\KeywordTok{rbernoulli}\NormalTok{(n_clientes, p)}
\NormalTok{  monto <-}\StringTok{ }\KeywordTok{rnorm}\NormalTok{(n_clientes, mu, sigma) }\OperatorTok{*}\StringTok{ }\NormalTok{reclamacion}
  \KeywordTok{sum}\NormalTok{(monto)}
\NormalTok{\}}

\KeywordTok{replicate}\NormalTok{(}\DecValTok{10000}\NormalTok{, }\KeywordTok{simular_mes}\NormalTok{()) }\OperatorTok
\StringTok{  }\KeywordTok{map_lgl}\NormalTok{(}\ControlFlowTok{function}\NormalTok{(x) x}\OperatorTok{>}\DecValTok{500000}\NormalTok{) }\OperatorTok
\StringTok{  }\KeywordTok{mean}\NormalTok{()}
\end{Highlighting}
\end{Shaded}

\begin{verbatim}
## [1] 0.9741
\end{verbatim}

\hypertarget{pregunta-11}{%
\subsection{Pregunta 11}\label{pregunta-11}}

Escribir una función para generar una mezcla de una distribución normal
multivariada con dos componentes con medias \(\mu_1\) y \(\mu_2\) y
matrices de covarianzas \(S_1\) y \(S_2\) respectivamente.

\begin{Shaded}
\begin{Highlighting}[]
\NormalTok{mixed_normal <-}\StringTok{ }\ControlFlowTok{function}\NormalTok{(n, mu_}\DecValTok{1}\NormalTok{, mu_}\DecValTok{2}\NormalTok{, S1, S2,p)\{}
  
\NormalTok{  m <-}\StringTok{ }\KeywordTok{matrix}\NormalTok{(}\KeywordTok{rnorm}\NormalTok{(n),}\DataTypeTok{nrow =} \KeywordTok{length}\NormalTok{(mu_}\DecValTok{1}\NormalTok{),}\DataTypeTok{ncol =}\NormalTok{n)}
\NormalTok{  mixed_normals <-}\StringTok{ }\KeywordTok{matrix}\NormalTok{(}\DecValTok{0}\NormalTok{, }\DataTypeTok{nrow=}\KeywordTok{length}\NormalTok{(mu_}\DecValTok{1}\NormalTok{), }\DataTypeTok{ncol =}\NormalTok{ n)}
  
\NormalTok{  B1 <-}\StringTok{ }\KeywordTok{chol}\NormalTok{(S1)}
\NormalTok{  B2 <-}\StringTok{ }\KeywordTok{chol}\NormalTok{(S2)}
  
  \ControlFlowTok{for}\NormalTok{( i }\ControlFlowTok{in} \DecValTok{1}\OperatorTok{:}\NormalTok{n)\{}
    \ControlFlowTok{if}\NormalTok{ (}\KeywordTok{runif}\NormalTok{(}\DecValTok{1}\NormalTok{) }\OperatorTok{<}\StringTok{ }\NormalTok{p)\{}
\NormalTok{      mixed_normals[,i] <-}\StringTok{ }\NormalTok{mu_}\DecValTok{1} \OperatorTok{+}\StringTok{ }\NormalTok{B1}\OperatorTok\NormalTok{m[,i]}
\NormalTok{    \}}
    \ControlFlowTok{else}\NormalTok{\{}
\NormalTok{      mixed_normals[,i] <-mu_}\DecValTok{2} \OperatorTok{+}\StringTok{ }\NormalTok{B2}\OperatorTok\NormalTok{m[,i]}
\NormalTok{    \}}
\NormalTok{  \}}
  
  \KeywordTok{return}\NormalTok{ (mixed_normals)}
\NormalTok{\}}
\end{Highlighting}
\end{Shaded}

Con el programa, generar una muestra de tamaño \(n = 1000\)
observaciones de una mezcla 50\% de una normal 4 dimensional con
\(\mu_1 =(0,0,0,0)\) y \(\mu_2 = (2, 3, 4, 5)\), y matrices de
covarianzas \(S_1 = S_2 = I_4\).

\begin{Shaded}
\begin{Highlighting}[]
\NormalTok{normal_multivariate <-}\StringTok{ }\KeywordTok{mixed_normal}\NormalTok{(}\DecValTok{10000}\NormalTok{,}\KeywordTok{c}\NormalTok{(}\DecValTok{2}\NormalTok{,}\DecValTok{3}\NormalTok{,}\DecValTok{4}\NormalTok{,}\DecValTok{5}\NormalTok{),}\KeywordTok{c}\NormalTok{(}\DecValTok{0}\NormalTok{,}\DecValTok{0}\NormalTok{,}\DecValTok{0}\NormalTok{,}\DecValTok{0}\NormalTok{),}\KeywordTok{diag}\NormalTok{(}\DecValTok{4}\NormalTok{),}\KeywordTok{diag}\NormalTok{(}\DecValTok{4}\NormalTok{),.}\DecValTok{5}\NormalTok{ )}
\end{Highlighting}
\end{Shaded}

Obtener los histogramas de las 4 distribuciones marginales.

\begin{Shaded}
\begin{Highlighting}[]
\KeywordTok{par}\NormalTok{(}\DataTypeTok{mfrow=}\KeywordTok{c}\NormalTok{(}\DecValTok{2}\NormalTok{,}\DecValTok{2}\NormalTok{))}
\ControlFlowTok{for}\NormalTok{ (i }\ControlFlowTok{in} \DecValTok{1}\OperatorTok{:}\DecValTok{4}\NormalTok{)\{}
  \KeywordTok{hist}\NormalTok{(normal_multivariate[i,],}\DataTypeTok{breaks=}\DecValTok{50}\NormalTok{)}
\NormalTok{\}}
\end{Highlighting}
\end{Shaded}

\includegraphics{sim-t3-sol_files/figure-latex/unnamed-chunk-34-1.pdf}

\hypertarget{pregunta-12-distribucion-de-wishart}{%
\subsubsection{Pregunta 12: Distribución de
Wishart}\label{pregunta-12-distribucion-de-wishart}}

Suponga que \(M=X'X\), donde \(X\) es una matriz de \(n \times d\) de
una muestra aleatoria de una distribución
\(\mathcal{N}_d(\mathbf{0}, \Sigma)\). Entonces
\(M \sim W_d(\Sigma, n)\). Notemos que en particular
\(W_1(\sigma ^2, n) \sim \sigma ^2 \chi^2_{(n)}\).

Una forma de generar observaciones de una distribución Wishart es
generar muestras de normales con la definición de arriba, pero notamos
que el método es muy cososo. Uno más eficiente se basa en la
descomposición ed Bartlett: sea \(T=(t_{ij})\) una matriz triangular
inferior de \(d \times d\) con entradas independientes que satisfacen

\begin{enumerate}
\def\labelenumi{\alph{enumi})}
\tightlist
\item
  \(t_{ij} \sim \mathcal{N}(0,1)\) independientes para \(i>j\)
\item
  \(t_{ii} \sim \sqrt{\chi^2_{(n-i+1)}}\) para \(i\in \{1, \cdots d\}\)
\end{enumerate}

Entonces, la matriz \(A=T'T\) es Wishart \(W_d(\mathbb{I}_n, d)\). Para
terminar, si \(\Sigma = LL'\) es la descomposición de Cholesky de
\(\Sigma\), entonces \(LAL' \sim W_d(\Sigma, d)\).

Compare el tiempo de ejecución de ambos métodos.

\begin{Shaded}
\begin{Highlighting}[]
\NormalTok{sg <-}\StringTok{ }\KeywordTok{matrix}\NormalTok{(}\KeywordTok{c}\NormalTok{(}\DecValTok{2}\NormalTok{,}\OperatorTok{-}\DecValTok{1}\NormalTok{,}\DecValTok{0}\NormalTok{,}\OperatorTok{-}\DecValTok{1}\NormalTok{,}\DecValTok{2}\NormalTok{,}\OperatorTok{-}\DecValTok{1}\NormalTok{,}\DecValTok{0}\NormalTok{,}\OperatorTok{-}\DecValTok{1}\NormalTok{,}\DecValTok{2}\NormalTok{), }\DataTypeTok{nrow=}\DecValTok{3}\NormalTok{)}

\NormalTok{rwishart_norm <-}\StringTok{ }\ControlFlowTok{function}\NormalTok{(}\DataTypeTok{df=}\DecValTok{1000}\NormalTok{, Sigma)\{}
\NormalTok{  X <-}\StringTok{ }\KeywordTok{mvrnorm}\NormalTok{(df, }\KeywordTok{rep}\NormalTok{(}\DecValTok{0}\NormalTok{, }\KeywordTok{dim}\NormalTok{(Sigma)[}\DecValTok{1}\NormalTok{]), Sigma)}
\NormalTok{  M <-}\StringTok{ }\KeywordTok{t}\NormalTok{(X) }\OperatorTok\StringTok{ }\NormalTok{X}
\NormalTok{\}}

\NormalTok{rwishart_norm_n <-}\StringTok{ }\ControlFlowTok{function}\NormalTok{(n, }\DataTypeTok{df=}\DecValTok{1000}\NormalTok{, Sigma)\{}
  \KeywordTok{replicate}\NormalTok{(n, }\KeywordTok{rwishart_norm}\NormalTok{(df, Sigma), }\DataTypeTok{simplify =}\NormalTok{ F)}
\NormalTok{\}}

\KeywordTok{tic}\NormalTok{(}\StringTok{'Método por normales'}\NormalTok{)}
\NormalTok{muestra_normal <-}\StringTok{ }\KeywordTok{rwishart_norm_n}\NormalTok{(}\DecValTok{1000}\NormalTok{,}\DecValTok{1000}\NormalTok{,sg)}
\KeywordTok{toc}\NormalTok{()}
\end{Highlighting}
\end{Shaded}

\begin{verbatim}
## Método por normales: 0.451 sec elapsed
\end{verbatim}

\begin{Shaded}
\begin{Highlighting}[]
\NormalTok{rwishart_choleksy <-}\StringTok{ }\ControlFlowTok{function}\NormalTok{(}\DataTypeTok{df=}\DecValTok{1000}\NormalTok{, d)\{}
  
\NormalTok{  tr_part <-}\StringTok{ }\DecValTok{0}\OperatorTok{:}\NormalTok{(d}\DecValTok{-1}\NormalTok{) }\OperatorTok
\StringTok{    }\KeywordTok{map}\NormalTok{(}\ControlFlowTok{function}\NormalTok{(i) }\KeywordTok{c}\NormalTok{(}\KeywordTok{rnorm}\NormalTok{(i), }\KeywordTok{rep}\NormalTok{(}\DecValTok{0}\NormalTok{, d}\OperatorTok{-}\NormalTok{i))) }\OperatorTok
\StringTok{    }\KeywordTok{unlist}\NormalTok{() }\OperatorTok
\StringTok{    }\KeywordTok{matrix}\NormalTok{(}\DataTypeTok{nrow=}\DecValTok{3}\NormalTok{, }\DataTypeTok{ncol=}\DecValTok{3}\NormalTok{, }\DataTypeTok{byrow=}\OtherTok{TRUE}\NormalTok{)}
  
\NormalTok{  diag_part <-}\StringTok{ }\DecValTok{1}\OperatorTok{:}\NormalTok{d }\OperatorTok
\StringTok{    }\KeywordTok{map}\NormalTok{(}\ControlFlowTok{function}\NormalTok{(i) }\KeywordTok{sqrt}\NormalTok{(}\KeywordTok{rchisq}\NormalTok{(}\DecValTok{1}\NormalTok{, df}\OperatorTok{-}\NormalTok{i}\OperatorTok{+}\DecValTok{1}\NormalTok{))) }\OperatorTok
\StringTok{    }\KeywordTok{unlist}\NormalTok{()}
\NormalTok{  diag_part <-}\StringTok{ }\NormalTok{diag_part }\OperatorTok{*}\StringTok{ }\KeywordTok{diag}\NormalTok{(d)}
  
\NormalTok{  A <-}\StringTok{ }\NormalTok{(tr_part}\OperatorTok{+}\NormalTok{diag_part) }\OperatorTok\StringTok{ }\KeywordTok{t}\NormalTok{(tr_part}\OperatorTok{+}\NormalTok{diag_part)}
  
  
\NormalTok{\}}

\NormalTok{rwishart_cholesky_n <-}\StringTok{ }\ControlFlowTok{function}\NormalTok{(n, }\DataTypeTok{df=}\DecValTok{1000}\NormalTok{, Sigma)\{}
\NormalTok{  d <-}\StringTok{ }\KeywordTok{dim}\NormalTok{(Sigma)[}\DecValTok{1}\NormalTok{]}
\NormalTok{  L <-}\StringTok{ }\KeywordTok{t}\NormalTok{(}\KeywordTok{chol}\NormalTok{(Sigma))}
  \KeywordTok{replicate}\NormalTok{(n, }\KeywordTok{rwishart_choleksy}\NormalTok{(df, d), }\DataTypeTok{simplify=}\OtherTok{FALSE}\NormalTok{) }\OperatorTok
\StringTok{    }\KeywordTok{map}\NormalTok{(}\ControlFlowTok{function}\NormalTok{(A) L }\OperatorTok\StringTok{ }\NormalTok{A }\OperatorTok\StringTok{ }\KeywordTok{t}\NormalTok{(L))}
\NormalTok{\}}

\KeywordTok{tic}\NormalTok{(}\StringTok{'Método de Bartlett'}\NormalTok{)}
\NormalTok{muestra_cholesky <-}\StringTok{ }\KeywordTok{rwishart_cholesky_n}\NormalTok{(}\DataTypeTok{n=}\DecValTok{1000}\NormalTok{,}\DataTypeTok{Sigma=}\NormalTok{sg)}
\KeywordTok{toc}\NormalTok{()}
\end{Highlighting}
\end{Shaded}

\begin{verbatim}
## Método de Bartlett: 0.518 sec elapsed
\end{verbatim}

En mi implementación es más rápido el método por normales, cosa que le
atribuyo a la vectorización que microoptimiza la ejecución en
\texttt{R}.

\hypertarget{pregunta-13}{%
\subsubsection{Pregunta 13}\label{pregunta-13}}

Las ocurrencias de huracanes que tocan tierra durante el fenómeno
meteorológico ``el Niño'' se modelan como un proceso Poisson (ver Bove
et al (1998)). Los autores aseguran que ``Durante un año de 'El Niño',
la probabilidad de dos o más huracanes haciendo contacto con tierra en
los estados Unidos es 28 \%''. Encontrar la tasa del proceso Poisson.

Si \(X \sim Po(\lambda)\), entonces
\(f(x;\lambda) = \frac{e^{-\lambda}\lambda^x}{x!}\).

Dado que \(P(X \geq 2) = .72\) tenemos que \(P(X \leq 1) = .28\) pero
\(P(X \leq 1) = e^{-\lambda} +e^{-\lambda}\lambda = .72\) y resolviendo
para \(\lambda\) obtenemos: \(\lambda = 1.043\)

De forma que la tasa del proceso buscado es \(\lambda = 1.043\)

\hypertarget{pregunta-14}{%
\subsection{Pregunta 14}\label{pregunta-14}}

Comenzando a medio día, comensales llegan a un restaurante de acuerdo a
un proceso poisson con tasa de 5 clientes por minuto. El tiempo que cada
cliente pasa comiendo es exponencial con media 40, y todos son
independientes entre sí y de los tiempos de arribo. Encuentre la
distribución, media y varianza del número de comensales en el
restaurante a las 2:00pm. Simule el restaurante para verificar.

Como los tiempos de consumo son exponenciales, el número de clientes que
sale del restaurante también es un proceso Poisson, independiente del de
llegada. Luego,

\[
N(t) = N_L(t)-N_S(t) \sim \mathrm{Poisson}(t(5-\frac{1}{40}))
\]

Por lo que \(N(120) \sim \mathrm{Poisson}(597)\) que tiene media y
varianza 597. Simulando para comparar

\begin{Shaded}
\begin{Highlighting}[]
\NormalTok{obtener_muestra <-}\StringTok{ }\ControlFlowTok{function}\NormalTok{()\{}
\NormalTok{  llegadas <-}\StringTok{ }\KeywordTok{poisson_nh}\NormalTok{(}\ControlFlowTok{function}\NormalTok{(t) }\DecValTok{5}\NormalTok{, }\DecValTok{5}\NormalTok{, }\DataTypeTok{Tmax=}\DecValTok{120}\NormalTok{)}
\NormalTok{  salidas <-}\StringTok{ }\KeywordTok{poisson_nh}\NormalTok{(}\ControlFlowTok{function}\NormalTok{(t) }\DecValTok{1}\OperatorTok{/}\DecValTok{40}\NormalTok{, }\DecValTok{1}\OperatorTok{/}\DecValTok{40}\NormalTok{, }\DataTypeTok{N=}\KeywordTok{length}\NormalTok{(llegadas), }\DataTypeTok{by_time =}\NormalTok{ F) }\OperatorTok{+}\StringTok{ }\NormalTok{llegadas[}\DecValTok{1}\NormalTok{]}
  \ControlFlowTok{if}\NormalTok{(}\KeywordTok{length}\NormalTok{(}\KeywordTok{which}\NormalTok{(salidas}\OperatorTok{<}\DecValTok{120}\NormalTok{))}\OperatorTok{==}\DecValTok{0}\NormalTok{)}
    \KeywordTok{length}\NormalTok{(llegadas)}
  \ControlFlowTok{else}
    \KeywordTok{length}\NormalTok{(llegadas) }\OperatorTok{-}\StringTok{ }\KeywordTok{max}\NormalTok{(}\KeywordTok{which}\NormalTok{(salidas}\OperatorTok{<}\DecValTok{120}\NormalTok{))}
\NormalTok{\}}

\NormalTok{muestra_restaurante <-}\StringTok{ }\KeywordTok{replicate}\NormalTok{(}\DecValTok{1000}\NormalTok{, }\KeywordTok{obtener_muestra}\NormalTok{())}

\KeywordTok{data_frame}\NormalTok{(}\DataTypeTok{sim=}\NormalTok{muestra_restaurante) }\OperatorTok
\StringTok{  }\KeywordTok{ggplot}\NormalTok{(}\KeywordTok{aes}\NormalTok{(}\DataTypeTok{sample=}\NormalTok{sim)) }\OperatorTok{+}
\StringTok{  }\KeywordTok{geom_qq_band}\NormalTok{(}\DataTypeTok{alpha=}\FloatTok{0.5}\NormalTok{, }\DataTypeTok{distribution=}\StringTok{'pois'}\NormalTok{, }\DataTypeTok{dparams=}\DecValTok{597}\NormalTok{) }\OperatorTok{+}
\StringTok{  }\KeywordTok{stat_qq_line}\NormalTok{() }\OperatorTok{+}
\StringTok{  }\KeywordTok{stat_qq_point}\NormalTok{() }\OperatorTok{+}
\StringTok{  }\KeywordTok{labs}\NormalTok{(}\DataTypeTok{x =} \StringTok{"Theoretical quantiles"}\NormalTok{, }\DataTypeTok{y =} \StringTok{"simulated values"}\NormalTok{)}
\end{Highlighting}
\end{Shaded}

\includegraphics{sim-t3-sol_files/figure-latex/unnamed-chunk-36-1.pdf}

\begin{Shaded}
\begin{Highlighting}[]
\NormalTok{probs <-}\StringTok{ }\KeywordTok{data_frame}\NormalTok{(muestra_restaurante) }\OperatorTok
\StringTok{  }\KeywordTok{group_by}\NormalTok{(muestra_restaurante) }\OperatorTok
\StringTok{  }\KeywordTok{count}\NormalTok{() }\OperatorTok
\StringTok{  }\KeywordTok{right_join}\NormalTok{(}\KeywordTok{data_frame}\NormalTok{(}\DataTypeTok{muestra_restaurante=}\DecValTok{1}\OperatorTok{:}\KeywordTok{max}\NormalTok{(muestra_restaurante))) }\OperatorTok
\StringTok{  }\KeywordTok{mutate}\NormalTok{(}\DataTypeTok{probs =} \KeywordTok{ifelse}\NormalTok{(}\KeywordTok{is.na}\NormalTok{(n), }\DecValTok{0}\NormalTok{, n)) }\OperatorTok
\StringTok{  }\KeywordTok{pull}\NormalTok{(probs)}
\end{Highlighting}
\end{Shaded}

\begin{verbatim}
## Joining, by = "muestra_restaurante"
\end{verbatim}

\begin{Shaded}
\begin{Highlighting}[]
\NormalTok{muestra_unique <-}\StringTok{ }\KeywordTok{table}\NormalTok{(muestra_restaurante) }\OperatorTok
\StringTok{  }\KeywordTok{names}\NormalTok{() }\OperatorTok
\StringTok{  }\KeywordTok{as.integer}\NormalTok{() }

\NormalTok{p <-}\StringTok{ }\KeywordTok{dpois}\NormalTok{(muestra_unique, }\DataTypeTok{lambda=}\DecValTok{597}\NormalTok{)}
\NormalTok{p <-}\StringTok{ }\NormalTok{p}\OperatorTok{/}\KeywordTok{sum}\NormalTok{(p)}

\KeywordTok{chisq.test}\NormalTok{(}\KeywordTok{table}\NormalTok{(muestra_restaurante), }\DataTypeTok{p=}\NormalTok{p)}
\end{Highlighting}
\end{Shaded}

\begin{verbatim}
## Warning in chisq.test(table(muestra_restaurante), p = p): Chi-squared
## approximation may be incorrect
\end{verbatim}

\begin{verbatim}
## 
##  Chi-squared test for given probabilities
## 
## data:  table(muestra_restaurante)
## X-squared = 158.69, df = 125, p-value = 0.02242
\end{verbatim}

\hypertarget{pregunta-15}{%
\subsubsection{Pregunta 15}\label{pregunta-15}}

Construyan un vector de 100 números crecientes y espaciados regularmente
entre 0.1 y 20. Llámenlo SIG2 . Ahora construyan otro vector de longitud
21 empezando en −1 y terminando en 1. Llámenlo RHO.

\begin{Shaded}
\begin{Highlighting}[]
\NormalTok{sig2 =}\StringTok{ }\KeywordTok{seq}\NormalTok{(}\DataTypeTok{from=}\NormalTok{.}\DecValTok{1}\NormalTok{,}\DataTypeTok{to=}\DecValTok{20}\NormalTok{,}\DataTypeTok{length.out=}\DecValTok{100}\NormalTok{)}
\NormalTok{rho =}\StringTok{ }\KeywordTok{seq}\NormalTok{(}\DataTypeTok{from=}\OperatorTok{-}\DecValTok{1}\NormalTok{,}\DataTypeTok{to=}\DecValTok{1}\NormalTok{,}\DataTypeTok{length.out=}\DecValTok{21}\NormalTok{)}
\end{Highlighting}
\end{Shaded}

Para cada entrada σ2 de SIG2 y cada entrada de RHO:

\begin{itemize}
\tightlist
\item
  Generar una muestra de tamaño N = 500 de una distribución bivariada
  normal \(Z =(X,Y)\) donde \(X \sim N(0,1)\) y \(Y \sim N(0,\sigma^2)\)
  y el coeficientede correlación de X y Y es \(\rho\). Z es una matriz
  de dimensiones 500 × 2.
\item
  Crear una matriz de 500 × 2, llámenlo EXPZ, con las exponenciales de
  las entradas de Z. ¿Qué distribución tienen estas variables
  transformadas?
\item
  Calculen el coeficiente de correlación, \hat{p} de las columnas de
  EXPZ. Grafiquen los puntos \((\sigma^2, \hat{ρ})\) y comenten sobre lo
  que obtuvieron.
\end{itemize}

La siguiente función permite generar una muestra aleatoria de tamaño n
de una variable aleatoria normal multivariada con vector de medias
\(\mu\) y matriz de covarianzas \(\Sigma\):

\begin{Shaded}
\begin{Highlighting}[]
\NormalTok{rnorm_multivariate <-}\StringTok{ }\ControlFlowTok{function}\NormalTok{(n,mu,Sigma)\{}
\NormalTok{  m <-}\StringTok{ }\KeywordTok{length}\NormalTok{(mu)}
  
\NormalTok{  eig <-}\StringTok{ }\KeywordTok{eigen}\NormalTok{(Sigma)}
\NormalTok{  lambda <-}\StringTok{ }\NormalTok{eig}\OperatorTok{$}\NormalTok{values; }
\NormalTok{  V <-}\StringTok{ }\NormalTok{eig}\OperatorTok{$}\NormalTok{vectors}
  
\NormalTok{  Q <-}\StringTok{ }\NormalTok{V }\OperatorTok\StringTok{ }\KeywordTok{diag}\NormalTok{(}\KeywordTok{sqrt}\NormalTok{(lambda)) }\OperatorTok\StringTok{ }\KeywordTok{t}\NormalTok{(V)}
  
\NormalTok{  Z <-}\StringTok{ }\KeywordTok{matrix}\NormalTok{(}\KeywordTok{rnorm}\NormalTok{(n}\OperatorTok{*}\NormalTok{m),}\DataTypeTok{nrow=}\NormalTok{n, }\DataTypeTok{ncol=}\NormalTok{m)}
\NormalTok{  X <-}\StringTok{ }\NormalTok{Z }\OperatorTok\StringTok{ }\NormalTok{Q }\OperatorTok{+}\StringTok{ }\KeywordTok{matrix}\NormalTok{(mu,n,m,}\DataTypeTok{byrow=}\NormalTok{T)}
  
  \KeywordTok{return}\NormalTok{ (X)}
\NormalTok{\}}
\end{Highlighting}
\end{Shaded}

Haciendo la simulación

\begin{Shaded}
\begin{Highlighting}[]
\NormalTok{normales=}\KeywordTok{list}\NormalTok{(}\OtherTok{NULL}\NormalTok{)}
\NormalTok{EXPZ =}\StringTok{ }\KeywordTok{list}\NormalTok{(}\OtherTok{NULL}\NormalTok{)}
\NormalTok{i<-}\DecValTok{1}
\NormalTok{sigmas <-}\StringTok{ }\KeywordTok{rep}\NormalTok{(}\DecValTok{0}\NormalTok{,}\DecValTok{2100}\NormalTok{)}
\NormalTok{rho_gorros <-}\StringTok{ }\KeywordTok{rep}\NormalTok{(}\DecValTok{0}\NormalTok{,}\DecValTok{2100}\NormalTok{)}
\NormalTok{rhos <-}\StringTok{ }\KeywordTok{rep}\NormalTok{(}\DecValTok{0}\NormalTok{,}\DecValTok{2100}\NormalTok{)}

\ControlFlowTok{for}\NormalTok{ (r }\ControlFlowTok{in}\NormalTok{ rho)\{}
  \ControlFlowTok{for}\NormalTok{ (s }\ControlFlowTok{in}\NormalTok{ sig2)\{}
    
\NormalTok{    Sigma =}\StringTok{ }\KeywordTok{matrix}\NormalTok{(}\KeywordTok{c}\NormalTok{(}\DecValTok{1}\NormalTok{,}\KeywordTok{sqrt}\NormalTok{(s)}\OperatorTok{*}\NormalTok{r,}\KeywordTok{sqrt}\NormalTok{(s)}\OperatorTok{*}\NormalTok{r,s),}\DataTypeTok{ncol=}\DecValTok{2}\NormalTok{)}
\NormalTok{    normales[[i]] <-}\StringTok{ }\KeywordTok{rnorm_multivariate}\NormalTok{(}\DecValTok{5000}\NormalTok{, }\KeywordTok{c}\NormalTok{(}\DecValTok{0}\NormalTok{,}\DecValTok{0}\NormalTok{), Sigma)}
\NormalTok{    EXPZ[[i]] <-}\StringTok{ }\KeywordTok{exp}\NormalTok{(normales[[i]])}
    
\NormalTok{    sigmas[i] <-}\StringTok{ }\NormalTok{s}
\NormalTok{    rho_gorros[i] <-}\StringTok{ }\KeywordTok{cor}\NormalTok{(EXPZ[[i]])[}\DecValTok{1}\NormalTok{,}\DecValTok{2}\NormalTok{]}
\NormalTok{    rhos[i] <-}\StringTok{ }\NormalTok{r}
    
\NormalTok{    i =}\StringTok{ }\NormalTok{i}\OperatorTok{+}\DecValTok{1}
\NormalTok{  \}}
\NormalTok{\}}
\end{Highlighting}
\end{Shaded}

EXPZ se obtiene tras aplicar la función exponencial a una variable
aleatoria distribuida normal de forma que EXPZ se distribuye lognormal.

Graficando:

\begin{Shaded}
\begin{Highlighting}[]
\KeywordTok{par}\NormalTok{(}\DataTypeTok{mfrow=}\KeywordTok{c}\NormalTok{(}\DecValTok{1}\NormalTok{,}\DecValTok{2}\NormalTok{))}
\KeywordTok{plot}\NormalTok{(sigmas,rhos)}
\KeywordTok{plot}\NormalTok{(sigmas,rho_gorros)}
\end{Highlighting}
\end{Shaded}

\includegraphics{sim-t3-sol_files/figure-latex/unnamed-chunk-40-1.pdf}

La gráfica del lado izquierda está formada por los puntos
\((\sigma^2, \rho)\) que fueron utilizados para generar las normales, la
gráfica del lado derecho está formada por puntos de la forma
\((\sigma^2, \hat{\rho})\) donde \(\hat{\rho}\) es la correlación entre
dos variables lognormales obtenidas a partir de las siguientes normales
\(N(0,1)\) y \(N(0,\sigma^2)\).

Notamos que la correlación cambió sustancialmente.


\end{document}
